\subsection{B-SAFE Risk Classification Framework}
\label{sec:methodology_risk_framework}
To provide a systematic and reproducible security assessment, we establish a formal risk classification framework. Each identified threat category is specified through a standardized schema that defines its preconditions, the system invariants it threatens, its canonical attack vector, applicable defense mechanisms, and quantitative risk metrics.

\subsubsection{Risk Category Specification Schema}
Each risk category R is formally defined by the tuple \( (P, I, S, C, M) \) where:
\begin{itemize}
    \item \textbf{P = \{p$_1$, p$_2$, ..., p$_n$\}} represents the set of preconditions that must hold for an attack to be feasible.
    \item \textbf{I = \{inv$_1$, inv$_2$, ..., inv$_m$\}} represents the set of core system invariants threatened by the attack.
    \item \textbf{S = (s$_1$ → s$_2$ → ... → s$_k$)} represents the canonical sequence of steps in the attack vector.
    \item \textbf{C = \{Prevention, Mitigation, Detection\}} represents the categories of defense mechanisms and controls.
    \item \textbf{M = (L, I, D, R)} represents the quantitative risk metrics for the category.
\end{itemize}

\subsubsection{Quantitative Risk Scoring}
The composite risk score \( R \) is calculated using a weighted formula based on Likelihood (L), Impact (I), and Detectability (D), each rated on a scale of 1 to 5. The formula is defined as:
\begin{equation}
    \text{Risk Score} = (w_L \times L) + (w_I \times I) - (w_D \times D)
\end{equation}
We adopt a qualitative-first stance for risk prioritization, aligning with NIST/ISO practice that treats risk primarily as a function of likelihood and impact. Detectability is modeled as a \textit{control-gap attribute} considered alongside (not subtracted from) the composite score. Where quantitative scoring is needed, we use a simple weighted sum of likelihood and impact; detectability is reported separately to guide monitoring and response investments. Sensitivity analysis examines ranking stability under weight variation.

\begin{figure}[H]
\centering
\includegraphics[width=0.5\textwidth]{../figure/fig3.png}
\caption{Visual representation of the (P, I, S, C, M) schema applied to SC-1 Reentrancy Attack. Right: illustrative sensitivity of composite risk to weight variation demonstrates prioritization stability when \(w_I\) remains dominant.}
\label{fig:schema_blueprint}
\end{figure}
