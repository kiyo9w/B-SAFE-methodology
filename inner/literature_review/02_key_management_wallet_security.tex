\subsection{Key Management and Wallet Security}
Effective key management—encompassing the generation, storage, distribution, and deletion of cryptographic keys—is the most critical element for ensuring the security of any blockchain-based system, as even robust protocols are rendered useless if their keys are poorly managed \cite{fumy1993}. In decentralized environments, private keys are the ultimate root of authority, controlling not just digital assets but also identity and access rights. This reality is often summarized by the axiom, "Not your keys, not your coins," but its implications extend far beyond currency to all forms of on-chain interaction \cite{yu2024}. User-facing tools, commonly known as "wallets," are the primary interface for managing these keys, making their security a linchpin for the integrity of user actions on the blockchain \cite{suratkar2020}. This review reveals that risks stem not only from inherent technical vulnerabilities but are also profoundly shaped by user perceptions and behaviors when interacting with blockchain applications.

\subsubsection{A Taxonomy of Vulnerabilities in Key Management}
A recent systematic review categorizes attacks on user key management tools into six primary groups. Among these, attacks targeting the layers closest to the user—including Memory \& Storage, Operating System, and Software Layer—are considered the most common and significant threats to the security of on-chain operations \cite{houy2023}.

\begin{itemize}
	\item \textbf{Memory and Storage}: Attacks at this layer focus on extracting cryptographic secrets directly from hardware. Random Access Memory (RAM) analysis can expose private keys, PINs, and seed phrases \cite{houy2023}. Studies have shown that many wallet applications store this sensitive information in unencrypted plaintext, creating a severe vulnerability. Furthermore, weak key generation methods like "brain wallets" (deriving keys from memorable phrases) are highly insecure due to low entropy and can be compromised rapidly through brute-force attacks, granting an attacker full control over a user's on-chain identity and assets \cite{houy2023}.
	
	\item \textbf{Operating Systems}: Platform-specific vulnerabilities, particularly on Android, are frequently exploited. The \textbf{clipboard hijacking attack} remains a prevalent threat, where an attacker replaces a destination address with their own during a copy-paste operation. This can redirect not only funds but also ownership of other digital assets. Another technique involves abusing Android's \textbf{Accessibility Mode}, which allows a malicious application to access all user interface events, compromising any data entered or displayed \cite{houy2023}.
	
	\item \textbf{Software Layer}: Implementation flaws within the wallet application itself are a major source of risk. Many desktop wallets offer open \textbf{Remote Procedure Call (RPC)} interfaces, which, if improperly configured, allow another application to impersonate the user and authorize malicious transactions on their behalf. The improper use of or reliance on flawed third-party libraries also introduces critical weaknesses. An alarming finding from the literature is the significant gap between known security best practices and their actual implementation by wallet developers \cite{houy2023}.
\end{itemize}

\subsubsection{User Behavior and Perception in Securing Blockchain Access}
Technical vulnerabilities are only part of the story. Recent qualitative studies demonstrate that user behavior and perception play a decisive role in how they secure their access to the blockchain, often in complex and counter-intuitive ways \cite{yu2024}.

\paragraph{"Don't Put All Your Eggs In One Basket": Risk Diversification and Contextual Choice.}
Experienced users rarely seek a single "best" key management solution. Instead, they actively employ a risk diversification strategy by using multiple wallets to segregate assets and isolate risk associated with different types of on-chain activities \cite{yu2024}. The choice of tool is highly contextual: mobile wallets are favored for simple interactions, while PC-based wallets (browser extensions) are preferred for complex operations like interacting with dApps. This is because larger screens facilitate careful verification of transaction data, and PC environments allow for third-party security extensions that can vet smart contract interactions before signing \cite{yu2024}.

\paragraph{The Evolving Trust Landscape: From Hardware to Smart Contracts.}
The classic distinction between "hot wallets" (online) and "cold wallets" (offline) remains a core security concept \cite{suratkar2020}. Hardware wallets, a form of cold storage, have long been considered the "gold standard" because they store keys on a dedicated, offline physical device \cite{suratkar2020}.

However, the latest research indicates a significant shift in user trust. Many users are moving \textbf{away from hardware wallets}, citing cumbersome user experiences and, more critically, emerging security concerns about manufacturers' policies. Controversial updates, such as Ledger's key recovery service, have fueled anxiety that a third party could potentially compromise the "self-custody" principle \cite{yu2024}. In response, there is a growing interest in \textbf{smart contract wallets}. Users are attracted to their programmable security features, such as social recovery and multi-signature authorization. Some now perceive these wallets as offering a level of security comparable to hardware wallets but with superior usability, prompting a migration of assets and trust \cite{yu2024}.

\paragraph{Social Cybersecurity and the Human Element of Trust.}
Social relationships are becoming an integral layer of key management. Features like "guardians" in smart contract wallets leverage social ties for account recovery, moving beyond purely technical solutions \cite{yu2024}. However, this introduces human-centric challenges. Users struggle with the social friction and trust calculations of appointing others to such a critical role. A unique security practice emerging from this is "self-guardianship," where users appoint their \textbf{own other devices} as guardians to benefit from the security model without navigating complex interpersonal trust issues \cite{yu2024}.