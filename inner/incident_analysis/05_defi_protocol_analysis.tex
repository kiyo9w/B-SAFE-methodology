\subsection{DeFi Protocol Layer Security Analysis}
\label{sec:results_defi_protocol}

% TODO: Team member to populate this section with:
% - PRO-1: Flash Loan Attack Analysis
% - PRO-2: Oracle Manipulation Analysis
% - PRO-3: Protocol Governance Analysis
% - Each following the standardized (P, I, S, C, M) framework

\subsubsection{Risk Category PRO-1: Flash Loan Enabled Attacks}

\paragraph{Formal Risk Specification}

\begin{itemize}
    \item \textbf{Preconditions (P):}
    \begin{itemize}
        \item \textbf{P1: Atomic Flash Loan Availability:} There is an atomic flash loan service that allows borrowing large amounts of uncollateralized assets and repaying them in the same transaction, for example Aave, dYdX.
        \item \textbf{P2: Sufficient On-chain Liquidity / Exploitable DEX reserves:} There is enough liquidity on the AMM/DEX for an attacker to cause significant price impact with one or a few large swaps. \cite{werner2022sok}
        \item \textbf{P3: Vulnerable Protocol Logic:} The protocol decides important operations (mint/borrow/withdraw) based on spot price or calculations that do not take slippage/TWAP into account. (e.g.,\ vaults/tranches that calculate “share price” based on spot pools). \cite{harvest2021}
        \item \textbf{P4: High Composability / Cross-contract interactions in a single transaction:} Protocol allows multiple contracts (DEX, oracle, vault) to be called in the same transaction without temporal guards.\cite{werapun2023faa}
        \item \textbf{P5: Inadequate emergency controls or monitoring:} Lack of circuit breakers, pause mechanisms or real-time anomaly detection systems.\cite{alhaidari2025protecting}
    \end{itemize}

    \item \textbf{Threatened System Invariants (I):}
    \begin{itemize}
        \item \textbf{INV-ACCT (Conservation of Value / Accounting):} The total value reported by the protocol (TVL, pool share value) must be equal to the actual number of tokens on the contract and related external assets. (Sample predicate: $\mathrm{TotalValueReported}(t) = \Sigma\, \mathrm{tokenBalances}(\mathrm{contract}, t) + \mathrm{ExternalAssets}(t)$). Violation occurs when attacker withdraws more than the actual value.
        \item \textbf{INV-COLL (No-Negative\_Equity / Collateralization):} For all loans $L$: $\mathrm{collateralValue}(L,t) \geq \mathrm{liquidationThreshold} \times \mathrm{borrowedValue}(L,t)$. Flash loans can cause collateralValue to be temporarily inflated/depreciated due to price manipulation in the same transaction.
        \item \textbf{INV-PRICE (Price Stability / Oracle Consistency):} The Oracle/price feed used for risk decision must be within $\Delta\%$ of the multi-source reference on window $W$. (Temporal invariant: $G(|\mathrm{OraclePrice}(t) - \mathrm{RefPrice}(t)|/\mathrm{RefPrice}(t) \leq \Delta)$).
        \item \textbf{INV-TEMP (Temporal / Atomicity):} Sensitive operations (deposit $\rightarrow$ borrow $\rightarrow$ withdraw) cannot be completed within the same block/transaction if they are based on spot price assumptions (formalized: $G(\text{if deposit}(tx) \wedge \text{borrow}(tx) \text{ within same block then invalid})$).
        \item \textbf{INV-SLIP (Liquidity / slippage Bound):} A swap volume $v$ must not cause the pool price to fluctuate beyond a level that the protocol does not account for; the protocol must evaluate the slippage bound $f(v)$ when using spot prices.
        \item \textbf{INV-COMP (Composability Invariant):} When A reads the state/price from B, it must ensure that B cannot be manipulated in the same transaction to invalidate A's invariant. \cite{werner2022sok, eskandari2021sok}
    \end{itemize}

    \item \textbf{Canonical Attack Sequence (S):}
    \begin{enumerate}
        \item \textbf{Borrow (flash loan):} Attacker borrows a large amount of token X (no collateral required) in the same transaction.
        \item \textbf{Manipulate / Use Liquidity:} Use X to swap/pump tokens on AMM, manipulate data sources (DEX reserves, oracle inputs). \cite{werner2022sok}
        \item \textbf{Trigger Vulnerable Logic:} Call protocol function (mint/borrow/withdraw) using manipulated value/collateral. \cite{werapun2023faa}
        \item \textbf{Extract Value:} Withdraw/appropriate assets beyond valid value (withdraw, drain pool).
        \item \textbf{Repay:} Pay flash loan in the same transaction; attacker gets net profit. (Key: all steps above happen atomically -- no time for arbitrage or oracle to update response).
    \end{enumerate}
\end{itemize}

\paragraph{Defense Mechanism Analysis}

\begin{itemize}
    \item \textbf{Prevention Controls:}
    \begin{itemize}
        \item \textbf{P-C1 (Time-weighted/Windowed Pricing):}
            \begin{itemize}
                \item \textit{Mechanism:} Use TWAP (time-weighted average price) or windowed aggregation instead of spot price for every significant decision (collateral valuation, minting). \cite{werner2022sok}
                \item \textit{Parameters:} Window's length $W$ (for example 5--30 minutes), sample granularity (for example, per block or per $N$ seconds).
                \item \textit{Trade-offs:} Reduces risk of nuclear attack but increases price latency $\rightarrow$ impacts UX/latency; does not protect against slow manipulation.
            \end{itemize}
        \item \textbf{P-C2 (Same-Block/Temporal Guards):}
            \begin{itemize}
                \item \textit{Mechanism:} Forbid/block risky operation pairs in the same block (e.g.,\ forbid deposit $\rightarrow$ borrow in same block) or require explicit multi-tx flows for sensitive ops.
                \item \textit{Parameters:} noSameBlock flag / modifier; minimal block gap $g$ (e.g.,\ $\geq 1$ block).
                \item \textit{Trade-offs:} Prevents atomic exploits but reduces composability, increasing friction for users (some legitimate use-cases are affected).
            \end{itemize}
        \item \textbf{P-C3 (Formalized Invariants \& Pre-deployment Verification):}
            \begin{itemize}
                \item \textit{Mechanism:} Declare invariants as assertions (predicates/temporal logic) and apply formal verification / SMT / static analysis to ensure invariants are not broken by atomic paths. \cite{alhaidari2025protecting, wu2024strengthening_defi}
                \item \textit{Parameters:} Coverage targets (e.g.,\ assertion coverage \%), formal toolchain (Certora/SMT/Z3), test scenarios (flash-loan attack patterns).
                \item \textit{Trade-offs:} High cost and time; requires expertise; additional runtime controls still needed.
            \end{itemize}
    \end{itemize}

    \item \textbf{Mitigation Controls:}
    \begin{itemize}
        \item \textbf{M-C1 (Circuit Breakers / Emergency Pause):}
            \begin{itemize}
                \item \textit{Mechanism:} Automatically pause (or allow operator pause) sensitive functions (borrowing/withdraw) when abnormal metrics (oracle jump, swap volume spike) are detected.
                \item \textit{Parameters:} Trigger thresholds: $\Delta_{oracle}$ (max allowed price jump), $\alpha$ (swap volume $> \alpha \times \text{poolLiquidity}$), pause duration $T_{pause}$.
                \item \textit{Trade-offs:} Effectively reduces damage immediately but creates centralization and requires ops/governance to resume.
            \end{itemize}
        \item \textbf{M-C2 (Dynamic Fees / Slippage Limiting):}
            \begin{itemize}
                \item \textit{Mechanism:} Apply dynamic fee multiplier or slippage limit when volume/speed exceeds threshold, reducing attack profit.
                \item \textit{Parameters:} Fee multiplier $f_m$ (e.g.,\ $\times 2$ -- $\times 10$), max slippage \% enforced, volatility window.
                \item \textit{Trade-offs:} May reduce legitimate activity during volatile periods; attacker can still pay fees if profit is high.
            \end{itemize}
        \item \textbf{M-C3 (Value-Dependent Multi-Tx Settlement / Adaptive Confirmations):}
            \begin{itemize}
                \item \textit{Mechanism:} Value-sensitive rules: large value withdrawals/moves require additional confirmations/time delay or multi-step confirmation (e.g.,\ on-chain timelock).
                \item \textit{Parameters:} confirmation count $c = \alpha \cdot \log_2(\text{value}/\$1000)+\beta$ (tuneable), governance timelock $\Delta_{gov}$.
                \item \textit{Trade-offs:} Increases transaction costs, may reduce UX; increases recovery time/cash-flow.
            \end{itemize}
    \end{itemize}

    \item \textbf{Detection Controls:}
    \begin{itemize}
        \item \textbf{D-C1 (Real-time Transaction Pattern Monitoring/Flash-loan Signatures):}
            \begin{itemize}
                \item \textit{Mechanism:} On-chain monitoring bots detect single-tx large borrow $\rightarrow$ swap $\rightarrow$ withdraw $\rightarrow$ repay patterns, abnormal swap sizes vs poolLiquidity, rapid state shifts. \cite{werapun2023faa}
                \item \textit{Parameters:} Thresholds $v > \alpha \times \text{poolLiquidity}$, pattern match rules, sampling window.
                \item \textit{Effectiveness:} Early detection but no blocking of mined transaction; need automation for quick response (pause).
            \end{itemize}
        \item \textbf{D-C2 (Cross-Source Price Divergence Alerts):}
            \begin{itemize}
                \item \textit{Mechanism:} Compare prices between multiple oracles/DEXs/CEXs; raise alarm if divergence $> \Delta$ within window $W$. \cite{eskandari2021sok}
                \item \textit{Parameters:} Source set size $n$ (recommend $\geq 3$), deviation threshold $\Delta\%$, sampling cadence.
                \item \textit{Effectiveness:} Effective in detecting manipulation on single feed; false positives when the market is volatile.
            \end{itemize}
        \item \textbf{D-C3 (Pre-Execution Simulation \& Mempool Analysis):}
            \begin{itemize}
                \item \textit{Mechanism:} Simulate mempool transaction sequences/use pre-execution analysis in relayer/frontends to evaluate potential price impacts before broadcast; flag risky transaction. \cite{alhaidari2025protecting}
                \item \textit{Parameters:} Simulation depth, acceptable latency budget, integration point (frontend/relayer/node).
                \item \textit{Effectiveness:} Cannot prevent miner-included transactions; adds latency \& infra cost; miners/MEV actors can bypass.
            \end{itemize}
    \end{itemize}
\end{itemize}

\paragraph{Empirical Incident Analysis}

\begin{itemize}
    \item \textbf{Case Study PRO-1.1: bZx Flash-loan Exploit (Feb 2020)}
    \begin{itemize}
        \item \textbf{Incident Classification:}
            \begin{itemize}
                \item \textit{Precondition Analysis:} P1\checkmark (flash loans used), P2\checkmark (sufficient liquidity on Uniswap for price impact), P3\checkmark (bZx relied on pricing logic vulnerable to slippage), P4\checkmark (composability exploited). \cite{foxley2020flashloan}
                \item \textit{Sequence \& violation:} Attacker took a large flash loan, manipulated price on an AMM and exploited bZx's margin/payout logic in the same transaction, violating INV-COLL and INV-ACCT. \cite{foxley2020flashloan}
            \end{itemize}
        \item \textbf{Quantitative Impact Assessment:}
            \begin{itemize}
                \item \textit{Direct Losses:} Reported series of bZx incidents produced losses in the order of hundreds of thousands to millions across multiple events (initial Feb 2020 incidents documented in press/postmortem).
                \item \textit{Systemic Effects:} Reputation damage, protocol freezes and emergency fixes; triggered community emphasis on TWAP and temporal guards.
            \end{itemize}
        \item \textbf{Counterfactual Analysis:}
            \begin{itemize}
                \item \textit{Prevention:} If bZx had used TWAP or temporal guards (disallow same-block borrow/use), the atomic exploit vector would have been closed. \cite{werner2022sok}
                \item \textit{Mitigation:} A circuit breaker triggered by anomalous swap volumes or price jumps could have halted withdrawals.
                \item \textit{Detection:} On-chain monitoring (catching large single-transaction swap signatures) would have raised alerts earlier.
            \end{itemize}
    \end{itemize}
        \item \textbf{Case Study PRO-1.2: Harvest Finance Exploit (Oct 2020)}
    \begin{itemize}
        \item \textbf{Incident Classification:}
            \begin{itemize}
                \item \textit{Precondition Analysis:} P1\checkmark (attacker used flash interactions), P2\checkmark (certain Curve pools/underlying liquidity allowed manipulation), P3\checkmark (Harvest's share-price computation trusted pool state without conservative slippage accounting). \cite{harvest2021, khatri2020harvest}
                \item \textit{Sequence \& violation:} Attacker used large flash-loan swaps to distort pool ratios feeding Harvest vaults, then withdrew inflated USD value $\rightarrow$ violated INV-ACCT and INV-SLIP.
            \end{itemize}
        \item \textbf{Quantitative Impact Assessment:}
            \begin{itemize}
                \item \textit{Direct Losses:} $\approx$\$24M (widely reported), attacker used multi-swap pattern to extract value. \cite{khatri2020harvest, thompson2020harvest}
            \end{itemize}
        \item \textbf{Counterfactual Analysis:}
            \begin{itemize}
                \item \textit{Prevention:} Use of TWAP or multi-source valuation for vault share pricing would have prevented instantaneous spot manipulation. \cite{werner2022sok}
                \item \textit{Mitigation:} Dynamic slippage limits and auto-pause on abnormal swaps would have limited extracted value.
                \item \textit{Detection:} Pattern detectors recognizing single-tx, high-volume swap sequences could have triggered emergency pause.
            \end{itemize}
    \end{itemize}
\end{itemize}


\subsubsection{Risk Category PRO-2: Oracle/Price-Feed Manipulation}

\paragraph{Formal Risk Specification}

\begin{itemize}
    \item \textbf{Preconditions (P):}
        \begin{itemize}
            \item \textbf{P1: Reliance on manipulable price/data feeds:} The protocol relies on one or more price/signal sources that can be influenced (e.g.,\ prices from a DEX spot pair, a CEX API, or a centralized oracle). \cite{chainalysis2023oracle}
            \item \textbf{P2: Low liquidity or cheap manipulation vector on feed sources:} Data sources (AMM pools, orderbooks, CEX snapshots) have low liquidity depth enough for attackers (with their own capital or flash-loans) to manipulate prices. \cite{kessler2022exploit}
            \item \textbf{P3: Immediate use of raw feed for high-value operations:} Protocol uses the raw feed price to perform actions with high financial consequences (e.g.,\ margin opening, collateral calculation, lending) without sanity checks / smoothing. \cite{chainlink2021defi}
            \item \textbf{P4: Lack of multi-source aggregation/fallback:} Lack of multi-source aggregation/median/fallback when one source deviates. \cite{chainlink2021defi}
            \item \textbf{P5: Insufficient monitoring or automated halting mechanisms:} Lack of cross-market surveillance, divergence alarms, or the ability to automatically halt when price deviates significantly. \cite{chainalysis2023oracle}
        \end{itemize}

    \item \textbf{Threatened System Invariants (I):}
        \begin{itemize}
            \item \textbf{INV-PRICE (Oracle Price Accuracy):} At any time $t$ in window $W$, the oracle price must be within $\Delta\%$ of the multi-source reference price set (predicate: $|\text{OraclePrice}(t) - \text{RefMedianPrice}(t)|/\text{RefMedianPrice}(t) \leq \Delta$). Violation occurs when the oracle offers a price that is superior to the market. \cite{chainalysis2023oracle}
            \item \textbf{INV-COLL (Collateralization / No-Negative-Equity):} For each position/loan $L$, $\text{collateralValue}(L,t) \geq \text{liquidationThreshold} \times \text{borrowedValue}(L,t)$. If the oracle quotes the wrong price (too high or too low), this invariant can be broken, leading to bad debt / insolvency. \cite{kessler2022exploit}
            \item \textbf{INV-ARB (Arbitrage-Free / No-Spurious-Arbitrage):} The oracle should not create large price differences compared to other markets, which would create conditions for empty profit arbitrage (which an attacker could exploit to drain liquidity). \cite{solidus2022mango}
            \item \textbf{INV-LIVENESS (Protocol Solvency/Asset Liveness):} The protocol must maintain solvency and avoid falling into a negative-reserve state; oracle manipulation can trigger system losses and break this invariant. \cite{akartuna2022mango}
        \end{itemize}

    \item \textbf{Canonical Attack Sequence (S):}
        \begin{enumerate}
            \item \textbf{Upstream price manipulation:} Attacker buys/sells in large quantities or in batches to change the price at the data source (e.g.,\ pump MNGO on CEXs/DEXs or manipulate AMM pairs). \cite{solidus2022mango, kessler2022exploit}
            \item \textbf{Oracle read:} Protocol reads manipulated price (receives spot price / TWAP recent price if window is small).
            \item \textbf{Exploit protocol logic:} Based on the wrong price, attacker opens position, borrows, mints, or withdraws assets (some protocols allow withdrawing or opening margin immediately according to feed price).
            \item \textbf{Unwind/profit:} Attacker takes profit, can revert the market after withdrawing; protocol bears bad debt / reduces TVL. \cite{akartuna2022mango}
        \end{enumerate}
\end{itemize}

\paragraph{Defense Mechanism Analysis}
\begin{itemize}
    \item \textbf{Prevention Controls:}
        \begin{itemize}
            \item \textbf{P-C1 (Multi-source aggregated oracles):}
                \begin{itemize}
                    \item \textit{Mechanism:} Take data from $\geq 3$ independent sources (Chainlink, DEX TWAPs, CEX snapshots) and use an aggregator (median or trimmed mean) to calculate a reference price before using it for risk decisions. \cite{chainlink2021defi, chainalysis2023oracle}
                    \item \textit{Parameters:} Number of sources $n \geq 3$; aggregation method (median / trimmed mean); refresh cadence $\tau$ (e.g.,\ 1--30s).
                    \item \textit{Trade-offs:} Reduces the concurrency risk of single-source manipulation but increases oracle costs, update latency, and operational complexity.
                \end{itemize}
            \item \textbf{P-C2 (Time-weighted averaging with robust windows):}
                \begin{itemize}
                    \item \textit{Mechanism:} Use TWAP / geometric mean over a carefully chosen window $W$ to smooth transient spikes. Note that the window dimension needs to be large enough to avoid flash-manipulation. \cite{chainlink2021defi}
                    \item \textit{Parameters:} Window $W$ (recommend tuneable, e.g.,\ 5m--60m depending asset liquidity); sampling granularity; outlier removal policy.
                    \item \textit{Trade-offs:} Reduces spike risk but may slow rational response to price moves; recent research also warns that short TWAPs can be manipulated with multiple txs (flash/slow attacks) so appropriate window sizes are needed. \cite{bai2024ormer}
                \end{itemize}
            \item \textbf{P-C3 (Oracle sanity checks / bounded update policy):}
                \begin{itemize}
                    \item \textit{Mechanism:} Before accepting an update, check $|\text{newPrice} - \text{lastGoodPrice}| \leq \text{maxJumpPerc}$. If exceeded, reject or mark for manual/fallback. \cite{chainlink2021defi}
                    \item \textit{Parameters:} maxJumpPerc (e.g.,\ 10--30\% depending on asset volatility); fallback policy (use previous price, median of other sources, or pause).
                    \item \textit{Trade-offs:} Prevent large instantaneous jumps; Can block legitimate volatile markets (false positives), need careful tuning per-asset.
                \end{itemize}
        \end{itemize}

    \item \textbf{Mitigation Controls:}
        \begin{itemize}
            \item \textbf{M-C1 (Circuit breakers / automated halting of sensitive ops):}
                \begin{itemize}
                    \item \textit{Mechanism:} When divergence oracle vs reference $>$ threshold, automatically pause borrowing/withdrawals or freeze high-value actions. \cite{chainalysis2023oracle}
                    \item \textit{Parameters:} Divergence threshold $\Delta\%$, minimum window $W$ to confirm anomaly, pause duration $T_{pause}$.
                    \item \textit{Trade-offs:} Minimize immediate losses but create a centralized control point (governance/operator needed to resume) and may disrupt legitimate markets.
                \end{itemize}
            \item \textbf{M-C2 (Dynamic margin / emergency collateralization adjustments):}
                \begin{itemize}
                    \item \textit{Mechanism:} When abnormal feed is detected, increase margin requirements or force tighter liquidation parameters to protect the system.
                    \item \textit{Parameters:} Margin multiplier $m$ (e.g.,\ $\times 1.2$ -- $\times 2$), emergency liquidation fee/priority.
                    \item \textit{Trade-offs:} Limit risk but may liquidate valid user positions during real volatility.
                \end{itemize}
            \item \textbf{M-C3 (Insurance funds \& debt-absorption mechanisms):}
                \begin{itemize}
                    \item \textit{Mechanism:} Maintain protocol reserves to absorb bad debt; implement debt auctions/liquidator incentivization to handle bad debt.
                    \item \textit{Parameters:} Insurance size as \%TVL (e.g.,\ 0.5--5\%); auction parameters.
                    \item \textit{Trade-offs:} Capital-intensive, creates maintenance costs; does not prevent exploits but minimizes impact on depositors.
                \end{itemize}
        \end{itemize}
    \item \textbf{Detection Controls:}
        \begin{itemize}
            \item \textbf{D-C1 (Cross-market surveillance \& divergence alerts):}
                \begin{itemize}
                    \item \textit{Mechanism:} Continuously compare feed prices with CEX/DEX sources; raise alert if $|p_{feed} - p_{refMedian}| > \Delta$ within window $W$. \cite{chainalysis2023oracle}
                    \item \textit{Parameters:} Source set size $n$, deviation threshold $\Delta\%$, sampling cadence.
                    \item \textit{Trade-offs:} Early detection of manipulation on single feed; false positives in volatile markets.
                \end{itemize}
            \item \textbf{D-C2 (Orderbook \& on-chain trading pattern anomaly detection):}
                \begin{itemize}
                    \item \textit{Mechanism:} Monitor orderbook depth, sudden large buys/sells, and sequences of on-chain swaps that correlate with oracle updates; alert and optionally throttle affected operations. \cite{solidus2022mango}
                    \item \textit{Parameters:} Volume multipliers $\alpha$ (e.g.,\ $> \times 10$ typical depth), imbalance metrics.
                    \item \textit{Trade-offs:} Effectively detects market manipulation; requires access to orderbook data and infra.
                \end{itemize}
            \item \textbf{D-C3 (Pre-action checks for high-value ops):}
                \begin{itemize}
                    \item \textit{Mechanism:} For operations above threshold value, require operator review, multisig confirmation, or delay before execution if price volatility is high.
                    \item \textit{Parameters:} Value threshold $V_t$ for gating; review window $T_r$.
                    \item \textit{Trade-offs:} Increases safety for high-value ops but slows down normal operations and may cause centralization.
                \end{itemize}
        \end{itemize}
\end{itemize}

\paragraph{Empirical Incident Analysis}
\begin{itemize}
    \item \textbf{Case Study PRO-2.1: Mango Markets (Oct 2022)}
        \begin{itemize}
            \item \textbf{Incident Classification:}
                \begin{itemize}
                    \item \textit{Precondition Analysis:} Mango depends on price feeds and cross-market data; The attacker performed cross-market manipulation, buying large amounts of MNGO on many exchanges, causing the oracle-reported price to spike. \cite{solidus2022mango}
                    \item \textit{Sequence \& violation:} Attacker pumped MNGO price across exchanges (within $\approx$10 minutes) $\rightarrow$ Mango oracle reported inflated collateral values $\rightarrow$ attacker borrowed $\approx$\$116M causing protocol insolvency; violations: INV-PRICE, INV-COLL, INV-LIVENESS. \cite{akartuna2022mango}
                \end{itemize}
            \item \textbf{Quantitative Impact Assessment:}
                \begin{itemize}
                    \item \textit{Direct Losses:} Reported outflows/negative balance $\approx$ \$116--118M (numbers reported across analyses, attacker later returned some funds / legal outcomes changed later).
                    \item \textit{Indirect/systemic effects:} Sharp TVL decline on Solana; regulatory / forensic investigation; long-term reputational cost for Mango and on-chain margin trading.
                \end{itemize}
            \item \textbf{Counterfactual Analysis:}
                \begin{itemize}
                    \item \textit{Prevention:} If Mango had used robust multi-source aggregation + TWAP (longer window) and oracle sanity checks, the short aggressive pump would not have immediately inflated collateral value. \cite{chainlink2021defi}
                    \item \textit{Mitigation:} An automated circuit breaker on large divergence or requirement for multi-tx confirmation for high-value borrows could have limited extraction. \cite{chainalysis2023oracle}
                    \item \textit{Detection:} Cross-market surveillance that flagged the 2,300\% spike within minutes could have triggered emergency pause before borrow completion. \cite{solidus2022mango}
                \end{itemize}
        \end{itemize}
    \item \textbf{Case Study PRO-2.2: Inverse Finance (Apr 2022)}
        \begin{itemize}
            \item \textbf{Incident Classification:}
                \begin{itemize}
                    \item \textit{Precondition Analysis:} Inverse used a TWAP type oracle (Keep3r / SushiSwap pair) that was manipulable with relatively low capital; attacker injected funds into SushiSwap to inflate INV price as seen by oracle. \cite{kessler2022exploit}
                    \item \textit{Sequence \& violation:} Manipulate INV/ETH pair on SushiSwap $\rightarrow$ oracle reported inflated INV $\rightarrow$ attacker borrowed $\sim$\$15.6M across assets; violations: INV-PRICE, INV-COLL. \cite{kessler2022exploit}
                \end{itemize}
            \item \textbf{Quantitative Impact Assessment:}
                \begin{itemize}
                    \item \textit{Direct Losses:} Reported $\approx$ \$15.6M drained. \cite{kessler2022exploit}
                    \item \textit{Indirect effects:} Spotlight on fragility of DEX-based and short-window TWAP oracles; protocol response included incident response planning and oracle redesign.
                \end{itemize}
            \item \textbf{Counterfactual Analysis:}
                \begin{itemize}
                    \item \textit{Prevention:} Aggregating multiple sources (not relying predominantly on a single DEX pair) and increasing TWAP window or adding jump bounds would have raised cost of manipulation beyond attacker's capital. \cite{chainlink2021defi, bai2024ormer}
                    \item \textit{Mitigation:} Auto-pause on large divergence and insurance buffers could have reduced net losses.
                    \item \textit{Detection:} On-chain anomaly detection for the SushiSwap trades used in manipulation would have allowed faster operator intervention. \cite{kessler2022exploit}
                \end{itemize}
        \end{itemize}
\end{itemize}

\subsubsection{Risk Quantification}

Using our systematic scoring framework:

\begin{itemize}
    \item \textbf{PRO-FLASH-LOAN}
    \begin{itemize}
        \item \textbf{Likelihood (L = 4):} High. Flash loans are popular and many protocols still use spot price / lack temporal guards; many real world exploits. \cite{werapun2023faa, werner2022sok}
        \item \textbf{Impact (I = 4):} High. Losses range from several hundred thousand to tens of millions of dollars; can destroy the protocol's TVL. 
        \item \textbf{Detectability (D = 2):} Difficult. Atomic exploits are single-tx, difficult to detect before the tx is mined; detection usually happens post-process or requires a dedicated detection system. \cite{werapun2023faa}
        \item \textbf{Risk Score = 3.4}: $0.4 \times 4 + 0.5 \times 4 - 0.1 \times 2 = 1.6 + 2.0 - 0.2 = 3.4$ (High Priority)
    \end{itemize}
    
    \item \textbf{PRO-ORACLE/PRICE-FEED}
    \begin{itemize}
        \item \textbf{Likelihood (L = 4):} High. Oracle manipulation incidents are frequent where protocols rely on manipulable feeds; industry reports show a rise in oracle manipulation cases.\cite{chainalysis2023oracle}
        \item \textbf{Impact (I = 5):} High. Historical incidents (Mango $\approx$\$116M, aggregate hundreds of millions across events) demonstrate systemic potential for catastrophic loss. \cite{akartuna2022mango}
        \item \textbf{Detectability (D = 2):} Difficult. Manipulation can be rapid and precede detection; cross-market surveillance helps but may still be too late for instantaneous exploits. \cite{solidus2022mango}
        \item \textbf{Risk Score = 3.9}: $0.4 \times 4 + 0.5 \times 5 - 0.1 \times 2 = 1.6 + 2.5 - 0.2 = 3.9$ (Critical Priority)
    \end{itemize}
\end{itemize}