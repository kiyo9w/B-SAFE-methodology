\section{Discussion}
This section synthesizes the implications of the B-SAFE framework and the case studies for enterprise blockchain adoption. We reflect on the framework's effectiveness as a practical, checklist-driven instrument that elevates decision quality without imposing excessive technical burden.

\subsection*{Enterprise Implications}
The results suggest that disciplined application of B-SAFE improves readiness outcomes across business, technical, and operational domains. Key levers include early alignment on ROI, formalized platform selection criteria, strong key management (preferably HSM-backed), and pre-approved incident response playbooks integrated with the SOC.

\subsection*{Security Posture and Risk}
Security outcomes improve most when key management, smart contract audits, and API security are treated as first-class citizens within change management. Residual risk remains concentrated in third-party dependencies and cross-organizational governance; these require explicit contracts, SLAs, and shared telemetry.

\subsection*{Limitations}
This work emphasizes practical assessment over automated detection. Results depend on the accuracy of enterprise self-reporting, scope boundaries, and the maturity of participating teams. While the checklist is designed to be technology-agnostic, some controls are platform-sensitive and require tailoring.

\subsection*{Future Directions}
Future work should extend the checklist with sector-specific control profiles, add quantitative scoring calibration using more datasets, and explore lightweight automation for evidence collection and continuous compliance without reverting to heavyweight ML systems.