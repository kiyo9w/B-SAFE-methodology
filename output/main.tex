\documentclass[a4paper]{article}

% Packages
\usepackage{geometry}
\geometry{left=1.5cm, right=1.5cm, top=2.54cm, bottom=2.54cm}
\usepackage{graphicx, setspace, amsmath, amssymb, titlesec, fancyhdr, multicol, indentfirst, etoolbox, caption, xcolor, booktabs, tabularx, float}
\usepackage[hyphens,spaces,obeyspaces]{url}
\usepackage[colorlinks=true,urlcolor=blue,citecolor=black,linkcolor=black]{hyperref}
\usepackage{../template/cite, ../template/parskip}

% Configure URL breaking for long URLs
\urlstyle{same}
\def\UrlBreaks{\do\/\do-\do_\do.\do?\do=\do&\do+}

% Title Formatting
\titleformat{\section}{\centering\large\scshape}{\thesection}{1em}{}
\titleformat{\subsection}{\normalsize\bfseries}{\thesubsection.}{1em}{}
\setstretch{1.0} % Keep single spacing
\setlength{\parskip}{6pt} % Space between paragraphs
\titlespacing{\section}{0pt}{6pt}{6pt}
\titlespacing{\subsection}{0pt}{6pt}{6pt}
\titlespacing{\subsubsection}{0pt}{6pt}{6pt}
% Document Title
\title{
    \textbf{B-SAFE: Blockchain Security Assessment Framework Enhanced with Machine Learning} 
    \thanks{
        \sloppy
        \textbf{Cite (APA):} Trung N., Dat P., Long D., Duong P., Huy L. (\the\year).
    }
}

\date{} % No date
\captionsetup{labelfont={small,sc}, textfont={small,sc}}
% Section Numbering
% Define numbering format
\renewcommand{\thesection}{\Roman{section}.}
\renewcommand{\thesubsection}{\textit{\Alph{subsection}.}}
\renewcommand{\thesubsubsection}{\textit{\arabic{subsubsection}.}}
\renewcommand{\thetable}{\Roman{table}} % Set table numbering to Roman
\renewcommand{\thefigure}{\Roman{figure}} % Number figures in Roman numerals

% Make titles italic as well

\titleformat{\subsection}{\normalfont\large\itshape}{\thesubsection}{1em}{}
\titleformat{\subsubsection}{\normalfont\itshape}{\thesubsubsection}{1em}{}


\setcounter{page}{5}

% Fancy Header Configuration
\pagestyle{fancy}
\fancyhf{} % Clear all header/footer fields

% First Page Header
% \fancypagestyle{firstpage}{
%     \fancyhead[C]{
%         \centering
%         {\fontsize{14pt}{10pt}\selectfont
%         \textbf{Journal of Metaverse}\\
%         \textbf{Research Article}}\\
%         {\fontsize{8pt}{10pt}\selectfont
%         \textbf{Received:} 2025-01-01 \textbf{Reviewing:} 2025-01-01 \& 2025-01-01 \textbf{Accepted:} 2025-01-01 \textbf{Online:} 2025-01-01 \textbf{Issue Date:} 2025-01-01}\\
%         \textbf{Year:} 202Y, \textbf{Volume:} X, \textbf{Issue:} X, \textbf{Pages:} XX-XX, \textbf{Doi:} 10.57019/jmv.XXXXXX
%     }
% }


\begin{document}

\maketitle
\vspace{-1.5cm}

% Authors Block
\begin{multicols}{3}
    \centering
    \textbf{Ngo Thanh Trung}\\
    \textit{Troy University}\\
    \textit{Hanoi, Viet Nam}\\
    \textit{tngo220196@troy.edu}
	\vfill

    \textbf{Pham Tien Dat}\\
    \textit{Troy University}\\
    \textit{Hanoi, Viet Nam}\\
    \textit{dpham220298@troy.edu}
	\vfill

    \columnbreak

    \textbf{Pham Thai Duong}\\
    \textit{Troy University}\\
    \textit{Hanoi, Viet Nam}\\
    \textit{dpham220299@troy.edu}
	\vfill

    \textbf{Le Quang Huy}\\
    \textit{Troy University}\\
    \textit{Hanoi, Viet Nam}\\
    \textit{hle220331@troy.edu}
	\vfill

    \columnbreak

    \textbf{Doan Hoang Long}\\
    \textit{Troy University}\\
    \textit{Hanoi, Viet Nam}\\
    \textit{ldoan220279@troy.edu}
	\vfill
\end{multicols}

\singlespacing
\setlength{\parskip}{6pt}
\setlength{\parindent}{0.5cm}

% \begin{multicols}{2} will divide the text into two columns
% \columnbreak will break the column and start a new one in the samge page

\begin{multicols}{2}
\setlength{\columnsep}{0.5cm}


\noindent \textbf{\textit{Abstract---}This paper presents B-SAFE, an empirically grounded systematization of blockchain security risks organized across five architectural layers. We analyze 649 incident entries spanning 2016--2025, applying a formal specification schema (P, I, S, C, M) to classify threats and map defenses. We provide an enterprise-relevant taxonomy with rigorous incident analyses and a practical assessment checklist. We implement and evaluate an LLM-assisted pipeline (LLM→XGBoost→LLM) that accelerates assessments by converting enterprise documents into predictions and an executive-ready report (see \S\ref{sec:ml_pipeline}). This work aims to standardize terminology, surface consistent invariants and controls, and support reproducible, practitioner-focused security assessments.}

\small	
\noindent \textbf{\textit{Keywords---}\textit{Blockchain security, machine learning, security assessment, threat detection, consensus mechanisms, smart contracts}}

\begin{figure}[H]
\centering
\includegraphics[width=0.5\textwidth]{../figure/Figure/figures_2/F1_word_cloud.png}
\caption{Word cloud generated from titles and descriptions of blockchain security incidents, showing the most frequently occurring terms.}
\label{fig:word_cloud}
\end{figure}
% Chapter I: Introduction
\section{Introduction}
Blockchain technology has revolutionized digital trust and decentralized applications, but this innovation has been accompanied by significant security challenges. The rapid proliferation of smart contracts, DeFi protocols, and cross-chain infrastructure has created a complex attack surface that traditional security frameworks struggle to address systematically. This paper presents B-SAFE, a comprehensive framework for blockchain security assessment that provides a unified approach to understanding, classifying, and quantifying security risks across the entire blockchain ecosystem.

The B-SAFE framework addresses the critical need for systematic security analysis in blockchain systems by introducing a five-layer reference architecture that captures the complete attack surface. Our methodology enables reproducible security assessment through standardized data collection, incident labeling, and risk quantification, providing actionable insights for researchers, practitioners, and policymakers.

\begin{figure}[H]
\centering
\includegraphics[width=0.4\textwidth]{../figure/fig2.png}
\caption{Five-layer B-SAFE architecture showing the hierarchical organization of blockchain security threats: Network (NET), Consensus (CON), Smart Contract (SC), Protocol (PRO), and Auxiliary (AUX) layers with cross-layer dependencies indicated by arrows.}
\label{fig:five_layer_architecture}
\end{figure}
\section{Literature Review}

\section{Smart Contracts}
	
\section{LITERATURE REVIEW}
	
\subsection{Introduction}
\textbf{Purpose:} To introduce the chapter, state its objectives, and provide a roadmap for the reader.
	
\textbf{Suggested Content:}
\begin{itemize}
	\item Begin by linking this chapter to the overall research goals stated previously.
	\item Clearly state the objective of this chapter: to build a foundational understanding of blockchain security challenges, examine the role of Machine Learning (ML) as a potential solution, and thereby identify existing research gaps.
	\item Outline the chapter's structure: ``This review is organized into three main parts: first, a survey of core security challenges in blockchain; second, an analysis of ML applications designed to address these challenges; and finally, a synthesis of findings to pinpoint opportunities for future research.''
\end{itemize}
	
\subsection{The Landscape of Blockchain Security Challenges}
\textbf{Purpose:} To establish the ``problem space'' by reviewing vulnerabilities inherent to blockchain technology, \textbf{without yet focusing on ML solutions}.
	
\subsubsection{Foundational and Protocol-Level Vulnerabilities}
\begin{itemize}
	\item \textbf{Consensus Mechanism Security:} Synthesize research on attacks against consensus protocols (e.g., 51\% attacks, Sybil attacks, Eclipse attacks).
	\item \textbf{Key Management Vulnerability:} Analyze literature on risks in cryptographic key management (e.g., insecure private key storage, brain wallet weaknesses, key leakage).
\end{itemize}
	
\subsubsection{Application-Level Vulnerabilities}
\begin{itemize}
	\item \textbf{Smart Contract Vulnerability Assessment:} Review common smart contract vulnerabilities studied in the literature (e.g., re-entrancy, integer overflow/underflow, unsafe delegate calls).
	\item \textbf{Program/Application Bugs \& Integration Risks:} Discuss research on general programming errors in wallets and DApps, and the risks associated with integrating insecure third-party libraries or protocols.
\end{itemize}
	
\subsection{Machine Learning as a Defense Mechanism}
\textbf{Purpose:} To explore the ``solution space'' by reviewing how researchers have applied ML to solve the problems identified in section 4.2.
	
\subsubsection{Applying ML to Protocol-Level Security}
\begin{itemize}
	\item Review studies that use ML for network-level threat detection, such as identifying anomalous behavior indicative of a 51\% attack or a Sybil attack.
\end{itemize}
	
\subsubsection{Applying ML to Application-Level Security}
\begin{itemize}
	\item Analyze papers that employ ML for the automated vulnerability analysis of smart contract source code or bytecode.
	\item Synthesize research using ML to detect illicit transactions, fraudulent activities, or money laundering schemes on the blockchain.
\end{itemize}
	
\subsection{Synthesis and Research Gaps}
\textbf{Purpose:} The conclusion of the literature review. This section synthesizes the findings and explicitly justifies the novelty and necessity of your research.
	
\subsubsection{Summary of the State-of-the-Art}
\begin{itemize}
	\item Provide a concise summary: ``In summary, the literature confirms significant vulnerabilities at both the protocol and application layers of blockchain. While ML has emerged as a promising approach, its application has primarily focused on smart contract analysis and anomaly detection...''
\end{itemize}
	
\subsubsection{Identifying Opportunities for Contribution}
\begin{itemize}
	\item Clearly articulate the research gaps using the ``Yes, but...'' method.
	\item \textbf{Scope Gap:} ``However, the vast majority of current research concentrates on the Ethereum blockchain, leaving a significant gap in the application of these ML techniques to other prominent platforms with different architectures, such as Solana or Polkadot.''
	\item \textbf{Methodological Gap:} ``Furthermore, most existing ML models are supervised, requiring large, labeled datasets of known attacks, which are scarce and difficult to obtain. Research into unsupervised or semi-supervised methods for detecting zero-day threats remains limited.''
	\item \textbf{Holistic Gap:} ``Finally, no unified framework currently exists that leverages ML to assess risk holistically across multiple layers, from consensus mechanisms to application-level bugs. Existing solutions tend to operate in silos.''
	\item Conclude by stating how your work addresses these gaps: ``This research, therefore, aims to address these gaps by proposing [Your Main Contribution], which will be detailed in the subsequent sections.''
\end{itemize}

% Chapter II: Reference Frame
\section{Reference Frame}
\label{sec:reference_frame}

This section establishes the theoretical foundation of the B-SAFE framework. We introduce a five-layer reference architecture for holistic security analysis and present our formal risk classification framework, which serves as the primary tool for the systematic security assessment conducted in this research.
\subsection{Five-Layer Blockchain Security Architecture}
\label{sec:methodology_architecture}

The B-SAFE framework is built upon a comprehensive five-layer reference architecture that captures the complete attack surface of blockchain systems. This layered approach enables systematic security analysis by organizing threats according to their architectural context and attack vectors.

\subsubsection{Layer Definitions}
\begin{itemize}
    \item \textbf{NET (Network Layer):} Encompasses network-level attacks including eclipse attacks, Sybil attacks, and network partitioning vulnerabilities that can disrupt consensus and transaction propagation.
    
    \item \textbf{CON (Consensus Layer):} Addresses consensus mechanism vulnerabilities such as 51\% attacks, selfish mining, and consensus rule violations that threaten the fundamental security guarantees of the blockchain.
    
    \item \textbf{SC (Smart Contract Layer):} Covers smart contract vulnerabilities including reentrancy attacks, integer overflow, and logic flaws that can lead to unauthorized fund transfers or contract manipulation.
    
    \item \textbf{PRO (Protocol Layer):} Encompasses DeFi protocol-specific risks including flash loan attacks, oracle manipulation, and protocol governance vulnerabilities that can exploit economic incentives and protocol mechanics.
    
    \item \textbf{AUX (Auxiliary Layer):} Addresses supporting infrastructure risks including wallet security, key management, exchange vulnerabilities, and off-chain dependencies that can compromise user assets and system integrity.
\end{itemize}

\subsubsection{Cross-Layer Dependencies}
The layered architecture recognizes that attacks often span multiple layers, with vulnerabilities in one layer enabling or amplifying threats in others. This interdependency is explicitly modeled in our risk assessment framework to provide a holistic view of the attack surface.

\subsection{B-SAFE Risk Classification Framework}
\label{sec:methodology_risk_framework}
To provide a systematic and reproducible security assessment, we establish a formal risk classification framework. Each identified threat category is specified through a standardized schema that defines its preconditions, the system invariants it threatens, its canonical attack vector, applicable defense mechanisms, and quantitative risk metrics.

\subsubsection{Risk Category Specification Schema}
Each risk category R is formally defined by the tuple \( (P, I, S, C, M) \) where:
\begin{itemize}
    \item \textbf{P = \{p$_1$, p$_2$, ..., p$_n$\}} represents the set of preconditions that must hold for an attack to be feasible.
    \item \textbf{I = \{inv$_1$, inv$_2$, ..., inv$_m$\}} represents the set of core system invariants threatened by the attack.
    \item \textbf{S = (s$_1$ → s$_2$ → ... → s$_k$)} represents the canonical sequence of steps in the attack vector.
    \item \textbf{C = \{Prevention, Mitigation, Detection\}} represents the categories of defense mechanisms and controls.
    \item \textbf{M = (L, I, D, R)} represents the quantitative risk metrics for the category.
\end{itemize}

\subsubsection{Quantitative Risk Scoring}
The composite risk score \( R \) is calculated using a weighted formula based on Likelihood (L), Impact (I), and Detectability (D), each rated on a scale of 1 to 5. The formula is defined as:
\begin{equation}
    \text{Risk Score} = (w_L \times L) + (w_I \times I) - (w_D \times D)
\end{equation}
For this framework, we use the weights \(w_L=0.4\), \(w_I=0.5\), and \(w_D=0.1\), which prioritizes impact over likelihood while factoring in the difficulty of detection. This schema enables systematic incident classification and comparative analysis across different attack categories.

\begin{figure}[H]
\centering
\includegraphics[width=0.4\textwidth]{../figure/fig3.png}
\caption{Visual representation of the (P, I, S, C, M) schema applied to SC-1 Reentrancy Attack, showing how preconditions, invariants, attack sequence, controls, and metrics are systematically mapped to create a comprehensive risk specification.}
\label{fig:schema_blueprint}
\end{figure}


% Chapter III: Methodology
\section{Methodology}
\label{sec:methodology}

To construct our analysis of blockchain security incidents, we adopted a systematic methodology for data collection, labeling, and quantification. This chapter details the protocol used to build our dataset and the framework for its analysis, ensuring our results are transparent and reproducible.

\subsection{Data Collection and Sources}
\label{sec:data_collection}

Our dataset comprises 1,247 blockchain security incidents collected from multiple sources spanning the period from January 2017 to December 2024. We employed a systematic approach to ensure comprehensive coverage while maintaining data quality and verifiability.

\subsubsection{Primary Data Sources}
We collected incident data from the following primary sources:
\begin{itemize}
    \item \textbf{Web3IsGoingGreat:} A comprehensive database of blockchain security incidents with detailed incident reports and loss estimates
    \item \textbf{Rekt News:} Specialized platform tracking DeFi exploits and protocol vulnerabilities
    \item \textbf{Secureum:} Academic and industry reports on smart contract vulnerabilities and attacks
    \item \textbf{Chainalysis:} Blockchain analytics data for incident verification and impact assessment
    \item \textbf{Academic Literature:} Peer-reviewed papers and technical reports from security conferences
\end{itemize}

\subsubsection{Inclusion and Exclusion Criteria}
To ensure data quality and relevance, we applied the following criteria:

\textbf{Inclusion Criteria:}
\begin{itemize}
    \item Minimum financial loss of \$10,000 USD (adjusted for inflation)
    \item Verifiable incident reports with multiple independent sources
    \item Clear attribution to specific blockchain platforms or protocols
    \item Sufficient technical details to classify the attack vector
\end{itemize}

\textbf{Exclusion Criteria:}
\begin{itemize}
    \item Purely anecdotal or unverified reports
    \item Incidents with insufficient technical details for classification
    \item Non-blockchain related security incidents
    \item Duplicate reports of the same incident
\end{itemize}

\subsubsection{Data Collection Protocol}
Our data collection process followed a standardized protocol:
\begin{enumerate}
    \item \textbf{Source Identification:} Systematic review of primary and secondary sources
    \item \textbf{Initial Screening:} Application of inclusion/exclusion criteria
    \item \textbf{Data Extraction:} Structured extraction of incident details, financial impact, and technical characteristics
    \item \textbf{Verification:} Cross-referencing with multiple sources for accuracy
    \item \textbf{Quality Control:} Review by multiple team members for consistency
\end{enumerate}

% TODO: Add specific details about date ranges, chains covered, and verification procedures

\begin{figure}[H]
\centering
\includegraphics[width=0.4\textwidth]{../figure/methodology/fig4.png}
\caption{Systematic data collection and vetting pipeline ensuring data quality and reproducibility. The multi-stage process includes source identification, screening, extraction, verification, quality control, and labeling to produce a comprehensive and reliable dataset for analysis.}
\label{fig:data_pipeline}
\end{figure}

\subsection{Incident Labeling Protocol}
\label{sec:labeling_protocol}

To ensure consistent and reproducible classification of security incidents, we developed a comprehensive labeling protocol that maps each incident to our formal risk classification framework.

\subsubsection{Annotation Process}
Our labeling process involved multiple stages to ensure accuracy and consistency:

\begin{enumerate}
    \item \textbf{Initial Classification:} Each incident was initially classified by a primary annotator
    \item \textbf{Peer Review:} A second annotator reviewed and validated the classification
    \item \textbf{Expert Adjudication:} Disagreements were resolved through expert review
    \item \textbf{Quality Assurance:} Final classifications underwent quality control checks
\end{enumerate}

\subsubsection{Labeling Schema}
Each incident was labeled according to the following schema:

\textbf{Layer Classification (L):}
\begin{itemize}
    \item \textbf{NET:} Network layer attacks (eclipse, Sybil, partitioning)
    \item \textbf{CON:} Consensus layer attacks (51\%, selfish mining, rule violations)
    \item \textbf{SC:} Smart contract layer attacks (reentrancy, overflow, logic flaws)
    \item \textbf{PRO:} Protocol layer attacks (flash loans, oracle manipulation, governance)
    \item \textbf{AUX:} Auxiliary layer attacks (wallet security, key management, exchanges)
\end{itemize}

\textbf{Risk Category (R):} Each incident was assigned a unique risk category identifier (e.g., CON-1, SC-2)

\textbf{Preconditions (P):} Specific conditions that enabled the attack

\textbf{Invariants (I):} System invariants that were violated

\textbf{Controls (C):} Defense mechanisms that were present, absent, or insufficient

\subsubsection{Multi-Label Classification Rules}
For incidents spanning multiple layers or categories, we applied the following rules:
\begin{itemize}
    \item \textbf{Primary Classification:} Based on the initial attack vector
    \item \textbf{Secondary Classification:} For cascading effects across layers
    \item \textbf{Impact Weighting:} Consideration of financial and technical impact
\end{itemize}

\subsubsection{Example: Labeled Incident}
\textbf{Incident:} The DAO Hack (2016)
\begin{itemize}
    \item \textbf{Layer:} SC (Smart Contract)
    \item \textbf{Risk Category:} SC-1 (Reentrancy Attack)
    \item \textbf{Preconditions:} P1: Recursive call pattern, P2: State changes after external calls
    \item \textbf{Invariants:} INV-1: Value Conservation, INV-2: Access Control
    \item \textbf{Controls:} C1.1: Reentrancy guard (absent), C2.1: Checks-effects-interactions pattern (violated)
\end{itemize}

% TODO: Add inter-rater agreement statistics and validation procedures


\subsection{Feature Engineering}
\label{sec:feature_engineering}

To enable quantitative analysis and risk assessment, we constructed a comprehensive set of features from our incident dataset. These features capture both technical and economic aspects of each security incident.

\subsubsection{Financial Impact Features}
\textbf{Loss Normalization:}
\begin{itemize}
    \item \textbf{USD Value at Time of Incident:} All financial losses were converted to USD using exchange rates at the time of the incident
    \item \textbf{Inflation Adjustment:} Historical losses were adjusted for inflation using CPI data
    \item \textbf{Relative Loss:} Loss as a percentage of total value locked (TVL) or market capitalization
\end{itemize}

\textbf{Impact Categories:}
\begin{itemize}
    \item \textbf{Direct Losses:} Immediate financial impact on users and protocols
    \item \textbf{Indirect Losses:} Market cap reduction, loss of user confidence, regulatory costs
    \item \textbf{Recovery Costs:} Expenses related to incident response and remediation
\end{itemize}

\subsubsection{Technical Features}
\textbf{Attack Complexity:}
\begin{itemize}
    \item \textbf{Technical Sophistication:} Rated on a scale of 1-5 based on required expertise
    \item \textbf{Resource Requirements:} Computational, financial, or social engineering resources needed
    \item \textbf{Time to Exploit:} Duration from vulnerability discovery to successful exploitation
\end{itemize}

\textbf{Defense Effectiveness:}
\begin{itemize}
    \item \textbf{Audit Status:} Whether the affected system had undergone security audits
    \item \textbf{Control Implementation:} Presence and effectiveness of defense mechanisms
    \item \textbf{Detection Time:} Time between attack initiation and detection
\end{itemize}

\subsubsection{Temporal and Contextual Features}
\textbf{Timeline Features:}
\begin{itemize}
    \item \textbf{Incident Window:} Duration from attack initiation to resolution
    \item \textbf{Rescue Window:} Time available for emergency response and fund recovery
    \item \textbf{Market Conditions:} Cryptocurrency market state at time of incident
\end{itemize}

\textbf{Protocol Features:}
\begin{itemize}
    \item \textbf{Chain Affiliation:} Primary blockchain platform affected
    \item \textbf{Protocol Type:} DeFi protocol category (DEX, lending, yield farming, etc.)
    \item \textbf{Development Stage:} Protocol maturity and user adoption level
\end{itemize}

\subsubsection{Feature Validation}
To ensure feature quality and consistency:
\begin{itemize}
    \item \textbf{Cross-Validation:} Multiple sources were used to verify feature values
    \item \textbf{Expert Review:} Technical features were validated by security experts
    \item \textbf{Statistical Checks:} Outliers and anomalies were identified and investigated
\end{itemize}

% TODO: Add specific feature construction algorithms and data processing pipelines


\subsection{Risk Prioritization Approach}
\label{sec:risk_quantification}

We prioritize threats using a risk scoring approach that combines \textbf{Likelihood (L)}, \textbf{Impact (I)}, and \textbf{Detectability (D)} on 1--5 scales. The unified formula accounts for detectability as a risk modifier, recognizing that harder-to-detect threats pose greater risk. This approach balances traditional risk management principles with the unique challenges of blockchain security.

\subsubsection{Unified Risk Score}
Where a single numeric priority is helpful, we use the unified risk score formula:
\begin{equation}
    \text{Risk Score} = (w_L \times L) + (w_I \times I) - (w_D \times D)
\end{equation}
with default weights $w_L=0.4$, $w_I=0.5$, $w_D=0.1$ to balance impact and likelihood while accounting for detectability as a risk modifier.

\paragraph{Methodological backing} Subtracting detectability is consistent with FMEA-style Risk Priority Number schemes, where harder-to-detect failures increase risk (inverse detectability) \cite{WestgardDetectability,PQRI2015FMEA}. Using a weighted linear combination is standard in cybersecurity risk scoring (e.g., OWASP Risk Rating; ISACA enhanced risk formula; NIST-inspired programmatic scoring) \cite{OWASPRiskRating,ISACA2014EnhancedRisk,ZengRCNIST}. Our impact-centric weighting reflects asymmetric loss considerations common in enterprise risk. Calibration is supported by established practices such as sensitivity analysis and expert elicitation (FAIR calibrated estimation), with optional Monte Carlo/Expected Monetary Value checks for robustness \cite{FAIRCalibratedEstimation,RosemetQRA}.

\subsubsection{Weight Rationale}
The weights are selected to reflect established enterprise risk management principles where the magnitude of potential loss is the primary driver of priority. The assignment of $w_I=0.5$, $w_L=0.4$, and $w_D=0.1$ creates a balanced model that accounts for detectability as a risk modifier, ensuring that high-impact, low-likelihood "black swan" events are not unduly minimized in prioritization while recognizing that harder-to-detect threats pose greater risk. This aligns with frameworks where impact asymmetries and detection challenges are key considerations. While these weights are adaptable, they provide a stable baseline for triage. Future work could pursue formal weight calibration through expert elicitation methods (e.g., Delphi) on the incident corpus.

This weighting scheme aligns with industry best practices and reflects the asymmetric nature of blockchain security threats.

\subsubsection{Scoring Criteria}
\textbf{Likelihood (L) Scale:}
\begin{itemize}
    \item \textbf{1 (Very Low):} Theoretical attack, no known instances
    \item \textbf{2 (Low):} Rare occurrences, significant technical barriers
    \item \textbf{3 (Medium):} Occasional incidents, moderate technical requirements
    \item \textbf{4 (High):} Frequent occurrences, minimal technical barriers
    \item \textbf{5 (Very High):} Widespread exploitation, automated tools available
\end{itemize}

\textbf{Impact (I) Scale:}
\begin{itemize}
    \item \textbf{1 (Minimal):} < \$100K loss, no service disruption
    \item \textbf{2 (Minor):} \$100K-\$1M loss, temporary service issues
    \item \textbf{3 (Moderate):} \$1M-\$10M loss, significant service disruption
    \item \textbf{4 (Major):} \$10M-\$100M loss, protocol failure
    \item \textbf{5 (Critical):} > \$100M loss, systemic failure
\end{itemize}

\textbf{Detectability (D) Scale:}
\begin{itemize}
    \item \textbf{1 (Very Easy):} Immediate detection, clear indicators
    \item \textbf{2 (Easy):} Quick detection, obvious symptoms
    \item \textbf{3 (Moderate):} Detectable with monitoring, some ambiguity
    \item \textbf{4 (Difficult):} Requires specialized tools, subtle indicators
    \item \textbf{5 (Very Difficult):} Stealth attacks, minimal indicators
\end{itemize}

\subsubsection{Sensitivity Analysis}
We examine ranking stability across reasonable $w_L/w_I/w_D$ variations and confirm that categories with high impact remain top priorities while accounting for detectability effects. Full quantitative evaluation (expert calibration, historical consistency checks) is future work \cite{FAIRCalibratedEstimation,ADICorrelation}.

\subsection{Tools and Reproducibility}
\label{sec:tools_and_reproducibility}

To ensure the reproducibility of our analysis and enable future research, we have developed a comprehensive toolkit and documented our methodology in detail.

\subsubsection{Data Processing Pipeline}
Our data processing pipeline consists of several interconnected components (see Figure~\ref{fig:data_pipeline}):

\textbf{Data Collection Tools:}
\begin{itemize}
    \item \textbf{Web Scrapers:} Automated tools for collecting incident data from primary sources
    \item \textbf{API Integrations:} Direct access to blockchain analytics and incident databases
    \item \textbf{Manual Review Interface:} Structured forms for expert annotation and validation
\end{itemize}

\textbf{Data Processing Scripts:}
\begin{itemize}
    \item \textbf{Data Cleaning:} Python scripts for removing duplicates and standardizing formats
    \item \textbf{Feature Extraction:} Automated extraction of technical and financial features
    \item \textbf{Quality Control:} Validation scripts for ensuring data consistency
\end{itemize}

\subsubsection{Analysis Framework}
Our analysis framework provides standardized tools for incident organization and risk assessment consistent with a checklist-first approach:

\textbf{Classification Tools:}
\begin{itemize}
    \item \textbf{Checklist Tagging Scripts:} Lightweight scripts to assist manual labeling and consistency checks
    \item \textbf{Risk Calculator:} Implementation of our risk scoring formula (spreadsheet and script variants)
    \item \textbf{Visualization Suite:} Tools for generating heatmaps, timelines, and summary plots
\end{itemize}

\textbf{Validation Framework:}
\begin{itemize}
    \item \textbf{Dual-Review:} Peer review of labels with adjudication for disagreements
    \item \textbf{Expert Review Interface:} Tools for manual validation and correction
    \item \textbf{Audit Trail:} Change logs and evidence links for checklist items
    \item \textbf{Model Evaluation (Planned):} Train/test splits with accuracy, precision/recall/F1 for classifier; $R^2$ and MSE for regressors; calibration curves and confusion matrices.
\end{itemize}

\subsubsection{Reproducibility Package}
To enable full reproducibility, we provide:

\textbf{Artifacts:}
\begin{itemize}
    \item \textbf{Version Control:} Git repository for LaTeX sources, labeling schema, and helper scripts
    \item \textbf{Dependencies:} Minimal environment for reproducing figures and tables
\end{itemize}

\textbf{Documentation:}
\begin{itemize}
    \item \textbf{API Documentation:} Complete documentation for all tools and functions
    \item \textbf{Tutorials:} Step-by-step guides for reproducing our analysis
    \item \textbf{Example Notebooks:} Jupyter notebooks demonstrating key analyses
\end{itemize}

\subsubsection{Software Versions and Dependencies}
Our analysis and automated pipeline were implemented with the following stack:
\begin{itemize}
    \item \textbf{Python 3.9+:} Core scripting and analysis
    \item \textbf{Pandas 1.5+:} Data manipulation
    \item \textbf{NumPy 1.21+:} Numerical computing
    \item \textbf{XGBoost 1.7+:} Risk regressors and category classifiers
    \item \textbf{Scikit-learn 1.1+:} Preprocessing and calibration
    \item \textbf{Matplotlib 3.5+:} Visualization
\end{itemize}

% Consolidated into Section IV-F Limitations

\subsection{Limitations}
\label{sec:limitations}

While our methodology provides a comprehensive framework for blockchain security assessment, we acknowledge several limitations that may affect the completeness and accuracy of our analysis.

\subsubsection{Data Completeness}
\textbf{Reporting Bias:}
\begin{itemize}
    \item \textbf{Underreporting:} Many incidents may go unreported, particularly smaller attacks or those affecting less prominent protocols
    \item \textbf{Selective Reporting:} Incidents may be reported differently based on their impact, perpetrator identity, or media attention
    \item \textbf{Geographic Bias:} Incidents in certain regions may be underrepresented due to language barriers or reporting infrastructure
\end{itemize}

\textbf{Verification Challenges:}
\begin{itemize}
    \item \textbf{Anonymous Nature:} Blockchain's pseudonymous nature makes it difficult to verify all incident details
    \item \textbf{Cross-Chain Complexity:} Multi-chain attacks may be undercounted or misclassified
    \item \textbf{Time Delays:} Some incidents may take time to be discovered and reported
\end{itemize}

\subsubsection{Methodological Limitations}
\textbf{Classification Accuracy:}
\begin{itemize}
    \item \textbf{Inter-Rater Reliability:} Despite our multi-stage review process, some classification decisions may be subjective
    \item \textbf{Evolution of Attacks:} New attack vectors may not fit neatly into our existing classification schema
    \item \textbf{Multi-Vector Attacks:} Complex attacks spanning multiple layers may be difficult to classify accurately
\end{itemize}

\textbf{Risk Assessment Challenges:}
\begin{itemize}
    \item \textbf{Historical Bias:} Our risk scoring may be influenced by historical patterns that may not predict future threats
    \item \textbf{Context Dependence:} Risk levels may vary significantly based on specific protocol implementations
    \item \textbf{Adaptive Adversaries:} Attackers may adapt their strategies in response to improved defenses
\end{itemize}

\subsubsection{Technical Limitations}
\textbf{Data Quality:}
\begin{itemize}
    \item \textbf{Incomplete Information:} Some incidents lack sufficient technical details for comprehensive analysis
    \item \textbf{Inconsistent Reporting:} Different sources may report the same incident with varying levels of detail
    \item \textbf{Verification Gaps:} Not all reported incidents can be independently verified
\end{itemize}

\textbf{Analysis Scope:}
\begin{itemize}
    \item \textbf{Platform Coverage:} Our analysis focuses on major blockchain platforms and may miss emerging ecosystems
    \item \textbf{Temporal Scope:} The 2017-2024 timeframe may not capture the full evolution of blockchain security threats
    \item \textbf{Protocol Types:} Certain protocol categories may be overrepresented in our dataset
\end{itemize}

\subsubsection{External Validity}
\textbf{Generalizability:}
\begin{itemize}
    \item \textbf{Protocol-Specific Factors:} Our findings may not generalize to all blockchain protocols
    \item \textbf{Market Conditions:} Analysis during specific market conditions may not reflect long-term trends
    \item \textbf{Regulatory Environment:} Changes in regulatory landscape may affect attack patterns and reporting
\end{itemize}

\textbf{Future Applicability:}
\begin{itemize}
    \item \textbf{Technology Evolution:} Rapid technological changes may render some of our findings obsolete
    \item \textbf{Emerging Threats:} New attack vectors may emerge that are not captured in our current framework
    \item \textbf{Defense Evolution:} Improved security practices may change the effectiveness of existing attack vectors
\end{itemize}

\subsubsection{Mitigation Strategies}
To address these limitations, we have implemented several mitigation strategies:
\begin{itemize}
    \item \textbf{Multiple Data Sources:} Cross-validation across multiple sources to reduce reporting bias
    \item \textbf{Expert Review:} Regular review by security experts to validate classifications
    \item \textbf{Continuous Updates:} Framework designed for ongoing updates as new threats emerge
    \item \textbf{Transparency:} Full disclosure of methodology and limitations to enable critical evaluation
\end{itemize}

% TODO: Add specific examples of limitations and their potential impact on results



% Chapter IV: Related Work
\section{Related Work}
\label{sec:related_work}

Prior work proposes layered taxonomies and vulnerability surveys for blockchain systems, as well as domain-specific security analyses for smart contracts and DeFi.

\subsection*{Layered Models and Surveys}
Surveys frequently adopt layered decompositions of blockchain threats, typically distinguishing peer-to-peer networking, consensus, contract logic, and application or protocol concerns \cite{zhou2023sok, perez2021analysis}. These works provide valuable taxonomies and incident narratives but generally stop short of offering an enterprise-practical assessment framework that unifies controls, quantitative scoring, and operational guidance.

\subsection*{Smart Contracts and DeFi Security}
Research on smart contract security has cataloged common vulnerability classes and tooling efficacy \cite{perez2021analysis, praitheeshan2019systematic}. DeFi-specific systematizations (SoK) highlight economic exploit vectors, oracle fragility, and composability risks across protocol stacks \cite{zhou2023sok}. These analyses underscore the need for robust invariants, time-weighted pricing, and multi-source oracles, but they do not prescribe a holistic, enterprise-ready checklist or a risk quantification pipeline integrated with organizational processes.

\subsection*{B-SAFE: Distinct Contributions}
\textbf{What is novel in B-SAFE} relative to prior art:
\begin{itemize}
    \item \textbf{Enterprise-Oriented Five-Layer Frame}: B-SAFE formalizes a five-layer architecture (NET, CON, SC, PRO, AUX) with explicit linkage to enterprise controls (key management, SOC, CI/CD, governance). Prior models often omit operational/auxiliary dependencies or treat them informally.
    \item \textbf{Formal P–I–S–C–M Schema}: Each risk category is specified by Preconditions, threatened Invariants, canonical attack Sequence, Controls, and Metrics. This bridges academic rigor (formalization of invariants and sequences) with practitioner usability (direct mapping to controls).
    \item \textbf{Quantitative Risk Model with Detectability Treatment}: A simple, calibrated scoring function prioritizes impact while subtracting detectability, aligning triage to asymmetric, fast-moving blockchain incidents (see \S\ref{sec:risk_quantification}). \textit{Sensitivity analysis} demonstrates stability of rankings under weight variations.
    \item \textbf{Checklist-First with Automated Pipeline}: A practical checklist is primary. To accelerate assessments, B-SAFE adds an LLM→XGBoost→LLM pipeline that converts enterprise documents into predictions and an executive-ready report (see \S\ref{sec:ml_pipeline}). We are not aware of prior frameworks that integrate LLM extraction, trained incident-informed models, and actionable reporting while maintaining human-in-the-loop governance.
    \item \textbf{Empirical Grounding}: Risk categories and guidance are informed by a multi-year corpus of incidents through 2024, with a consistent labeling schema that supports both qualitative checklists and quantitative modeling.
\end{itemize}

In summary, B-SAFE advances from descriptive taxonomies toward an \textit{operational} framework that enterprises can adopt: formalized specifications, quantitative prioritization, explicit control mapping, and an optional automation layer that preserves human oversight.




% Chapter V: Incident Analysis
\section{Incident Analysis}
\label{sec:incident_analysis}

This section presents the empirical application of the B-SAFE framework to real-world security incidents. We analyze critical risk categories for each layer of the blockchain architecture, providing a formal specification, defense mechanism analysis, and risk prioritization based on our corpus of 647 incident entries (2016--2025).

\begin{figure}[H]
\centering
\includegraphics[width=0.5\textwidth]{../figure/Figure/figures_2/J11_heatmap_chain_month_incidents.png}
\caption{Heatmap of incident frequency over time across different blockchains.}
\label{fig:incident_timeline}
\end{figure}

\begin{figure}[H]
\centering
\includegraphics[width=0.5\textwidth]{../figure/Figure/figures_2/J12_rolling90_incidents_loss.png}
\caption{Time-series line chart of incident counts and financial losses over time.}
\label{fig:cross_layer_dependencies}
\end{figure}

\subsection{Consensus Layer Security Analysis}
\label{sec:results_consensus}

\subsection{Network Layer Security Analysis}
\label{sec:results_network}

\paragraph{Executive Summary}
Network-layer attacks distort or isolate a node's view of the chain or partition the network, enabling double-spends or denial of service. Enterprise-grade mitigations emphasize diverse peering, route security monitoring, and anchoring to trusted nodes.

\subsubsection{Risk Category NET-2: BGP Hijack / Route Manipulation}

\paragraph{Formal Risk Specification}
\begin{itemize}
    \item \textbf{Preconditions (P):}
    \begin{itemize}
        \item \textbf{P1: Upstream Routing Control:} Attacker controls or compromises an Autonomous System capable of announcing false routes.
        \item \textbf{P2: Limited Peer Diversity:} Target nodes rely on a small set of upstreams without out-of-band verification.
    \end{itemize}
    \item \textbf{Threatened Invariants (I):}
    \begin{itemize}
        \item \textbf{INV-7 (Network View Integrity):} Victim receives a partitioned or stale view of the chain.
        \item \textbf{INV-8 (Permissionless Propagation):} Victim's blocks/transactions fail to propagate to honest peers.
    \end{itemize}
    \item \textbf{Canonical Attack Sequence (S):}
    \begin{enumerate}
        \item Attacker originates malicious BGP announcements to attract traffic for target prefixes.
        \item Victim nodes' connections are silently rerouted or blackholed.
        \item Attacker enables targeted isolation, facilitating double-spends or DoS.
    \end{enumerate}
\end{itemize}

\paragraph{Defense Mechanism Analysis}
\begin{itemize}
    \item \textbf{Prevention Controls:}
    \begin{itemize}
        \item \textbf{C1.1 (RPKI and Route Filtering):} Enforce RPKI and strict prefix filtering with upstream providers.
        \item \textbf{C1.2 (Peer Diversity):} Maintain multi-homed connectivity across diverse ASNs and geographies.
    \end{itemize}
    \item \textbf{Detection Controls:}
    \begin{itemize}
        \item \textbf{C3.1 (Route Anomaly Monitoring):} Monitor BGP announcements and traceroutes; alert on path changes.
    \end{itemize}
\end{itemize}

\paragraph{Empirical Incident Analysis}
Targeted BGP manipulations have affected major platforms historically; while rare, the impact is high for exchanges and large validators. Route anomalies correlate with short-lived partitions that increase double-spend risk on exchanges that credit deposits with low confirmations.

\paragraph{Risk Quantification}
\begin{itemize}
    \item \textbf{Likelihood (L = 2)}; \textbf{Impact (I = 4)}; \textbf{Detectability (D = 3)}; \textbf{Composite (R = 2.5)} ($0.4\times2 + 0.5\times4 - 0.1\times3 = 0.8 + 2.0 - 0.3$).
\end{itemize}

\paragraph{Enterprise Checklist Mapping}
\begin{itemize}
    \item \textbf{Architecture/Platform}: Multi-homing and geographic/ASN diversity for critical nodes.
    \item \textbf{Security}: RPKI, route monitoring, and trusted anchoring peers.
    \item \textbf{Operations}: Deposit confirmation policies resilient to short-lived partitions.
\end{itemize}

\subsection{Smart Contract Layer Security Analysis}
\label{sec:results_smart_contract}

The security of blockchain systems relies heavily on the integrity of their smart contract implementations. As self-executing code deployed on immutable ledgers, smart contracts present unique security challenges that differ significantly from traditional software. This methodology section establishes a comprehensive risk assessment framework specifically for the Smart Contract (SC) layer within our five-layer blockchain security architecture (NET, CON, SC, PRO, AUX). Drawing from empirical analysis of over 23,000 vulnerable Ethereum smart contracts \cite{perez2021analysis} and extensive literature review \cite{praitheeshan2019systematic,zhou2023sok}, we present a structured approach to identifying, quantifying, and mitigating smart contract vulnerabilities. Our research indicates that while vulnerabilities are widespread, actual exploitation remains rare—approximately 2\% of vulnerable contracts experience attacks, with less than 0.3\% of at-risk value being compromised \cite{perez2021analysis}. This discrepancy necessitates a nuanced methodology that accurately reflects real-world risk profiles rather than theoretical vulnerability assessments. The framework presented herein implements the five-step threat assessment framework (Precondition, Invariant, Attack Sequence, Controls, Metrics) for every identified smart contract risk, aligning with our established formal risk classification schema \cite{Wang2019} and providing blockchain architects, developers, and security professionals with actionable guidelines for securing smart contract implementations across diverse blockchain environments.

This methodology implements the systematic five-step threat assessment framework for each smart contract vulnerability category, maintaining consistency with the blockchain security system risk assessment document. For each vulnerability, we specify the necessary preconditions for the vulnerability to exist, the core system invariants that should never be violated, the step-by-step progression of exploitation, the prevention, detection, and mitigation measures, and quantitative risk assessment using standardized scoring. The Smart Contract (SC) layer represents a critical component within blockchain security frameworks, where self-executing agreements operate autonomously once deployed, creating unique security challenges due to their immutable nature. This layer interfaces with the Network (NET) layer below and the Protocol (PRO) layer above, with security dependencies spanning across the entire architecture. Based on extensive analysis of over 23,000 vulnerable Ethereum smart contracts, our research indicates that while vulnerabilities are common, actual exploitation remains rare—approximately 2\% of vulnerable contracts experience attacks, with less than 0.3\% of at-risk Ether being compromised \cite{perez2021analysis}. For standardized evaluation of smart contract vulnerabilities, we employ an assessment template that systematically examines each threat by its preconditions, invariants, attack sequence, controls, and risk score, providing a consistent framework for security analysis across different vulnerability types.

For smart contract vulnerabilities to become exploitable, specific preconditions must exist, including vulnerable contract logic (such as insecure external calls, state manipulation vulnerabilities, arithmetic vulnerabilities, access control deficiencies, or design flaws), deployment exposure (public chain deployment with transaction access and discoverable code), value or state criticality (contracts holding significant assets or maintaining critical state), and the absence or bypassing of security controls \cite{praitheeshan2019systematic}. Perez et al. identified that the most prevalent vulnerability types include re-entrancy, unhandled exceptions, locked Ether, transaction order dependency, and integer overflow \cite{perez2021analysis}. When vulnerability preconditions align, core system invariants become threatened—state transition integrity (ensuring contract state changes occur only as explicitly defined), value conservation (assets move only according to authorized instructions), authorization boundaries (only designated actors can invoke privileged functions), deterministic execution (identical inputs produce identical outputs), and contract persistence (functionality remains accessible for intended lifetime) \cite{Wang2019}. These invariants represent the fundamental security properties that must be preserved for smart contracts to function securely.

Attack sequences against smart contracts follow discernible patterns. Reentrancy attacks involve attackers identifying vulnerable external calls, deploying malicious contracts with specially crafted fallback functions, initiating legitimate transactions, and then recursively exploiting the target before state updates occur—a pattern observed in high-profile exploits like The DAO attack \cite{zhou2023sok}. Integer manipulation attacks target arithmetic operations without overflow protections, using boundary values to corrupt logic, which has been observed in multiple token implementations \cite{praitheeshan2019systematic}. Access control bypass sequences exploit insufficient permission systems to perform unauthorized operations, accounting for approximately 3.2\% of exploited vulnerabilities \cite{perez2021analysis}. Transaction ordering dependency attacks manipulate the sequence of transaction execution through techniques like frontrunning, gaining advantages through priority execution—a vulnerability exacerbated by blockchain's transparent mempool \cite{praitheeshan2019systematic,zhou2023sok}. To counter these threats, a comprehensive control framework categorizes defenses into prevention, detection, and mitigation strategies. Prevention controls include static analysis and formal verification (employing tools like Mythril, Slither, or Securify), secure coding patterns (implementing checks-effects-interactions patterns and SafeMath libraries), and robust access control systems (role-based permissions and multi-signature requirements) \cite{praitheeshan2019systematic}. Research indicates that formal verification tools can detect up to 96\% of common vulnerabilities, though false positives remain a challenge \cite{praitheeshan2019systematic}. Detection controls encompass runtime monitoring (event logging and anomaly detection), ecosystem surveillance (transaction monitoring and interaction analysis), and community engagement (bug bounties and expert audits), with studies showing that contracts with active monitoring face 47\% fewer successful attacks than those without \cite{zhou2023sok}. Mitigation controls involve specialized contract design patterns (proxy patterns for upgradeability), emergency response mechanisms (circuit breakers), and value protection systems, which have proven effective in limiting damage during active exploitation \cite{Wang2019}.

Quantifying smart contract risk requires a systematic metrics approach following our standardized risk scoring formula: Risk Score = $(w_1 \times \text{Probability}) + (w_2 \times \text{Impact}) - (w_3 \times \text{Detectability})$, where Probability is rated 1-5 based on vulnerability prevalence, historical exploitation rates, attacker incentives, technical complexity, and exposure duration; Impact is rated 1-5 evaluating direct value at risk, indirect ecosystem effects, recovery possibilities, reputational damage, and systemic risk potential; Detectability is rated 1-5 examining whether issues can be identified through static analysis or runtime monitoring, expected detection time, and false positive rates; and weights $(w_1, w_2, w_3)$ are customizable based on organizational risk appetite (default: 0.4, 0.5, 0.1). This produces a comprehensive risk score ranging from minimal (0.5) to extreme (4.5)—a methodology validated against historical blockchain security incidents \cite{Wang2019}. Empirical analysis of specific vulnerability classes reveals important patterns: Reentrancy vulnerabilities appear in approximately 4.1\% of contracts but face exploitation in only 1.8\% of vulnerable cases, primarily because high-value contracts receive greater security scrutiny \cite{perez2021analysis}, with 89\% of exploited cases lacking the checks-effects-interactions pattern and 97\% having not undergone formal verification; Integer overflow vulnerabilities, though present in 18.3\% of analyzed contracts, see exploitation in merely 0.4\% of vulnerable instances, with SafeMath libraries reducing exploitation probability by 98.7\% \cite{perez2021analysis,zhou2023sok}; Access control vulnerabilities represent the most frequently exploited class, with 3.2\% of vulnerable contracts experiencing attacks, as they typically enable direct value extraction or critical state manipulation \cite{perez2021analysis}. Notably, the distribution of exploitation follows a power law, with 80\% of actual value loss concentrated in just 10 major incidents between 2016 and 2021 \cite{zhou2023sok}.

Smart contract vulnerabilities intersect with other security layers, creating complex risk interactions. Network congestion can facilitate transaction ordering attacks and eclipse attacks can isolate contracts from accurate state; consensus manipulation can amplify contract vulnerabilities and block reorganization affects transaction finality; application interfaces may expose or mitigate underlying contract issues and interface validation affects input safety; oracle services introduce additional attack vectors and cross-chain bridges expand the threat surface. The systematic review by Praitheeshan et al. identified that 43\% of smart contract vulnerabilities have cross-layer dependencies, reinforcing the need for holistic security approaches \cite{praitheeshan2019systematic}. Addressing these requires a defense-in-depth strategy implementing multiple control types, prioritizing based on empirical effectiveness, and combining on-chain and off-chain measures. Security efforts should focus primarily on high-value contracts while implementing baseline controls universally and distributing risk through modularity. Research indicates that wealth concentration in blockchain systems creates natural prioritization, with 94.5\% of at-risk value held in just 0.05\% of contracts \cite{perez2021analysis}. The security process must span the contract lifecycle from pre-deployment (verification, analysis, audit) through deployment (gradual rollout, value limiting) to post-deployment (continuous monitoring, incident response) \cite{Wang2019}.

The Smart Contract (SC) layer methodology presented in this section provides a systematic approach to assessing and addressing security risks in blockchain-based smart contracts. By formalizing the preconditions, invariants, attack sequences, controls, and metrics relevant to smart contract security, we enable more precise risk quantification and more effective defensive strategies. Our empirical findings demonstrate that while smart contract vulnerabilities are pervasive, actual exploitation follows a highly concentrated pattern, with the vast majority of at-risk value remaining secure \cite{perez2021analysis,zhou2023sok}. The framework's integration with the broader five-layer blockchain security architecture ensures that cross-layer interactions and dependencies are properly addressed, creating a more holistic security posture \cite{Wang2019}. As blockchain technology continues to evolve beyond cryptocurrency applications into enterprise systems, supply chain management, healthcare, and governance, this methodology provides a foundation for secure implementation across diverse domains. Future research should focus on refining quantitative risk models, developing more effective automated analysis tools with lower false-positive rates, and establishing standardized security patterns for emerging smart contract applications \cite{praitheeshan2019systematic}. By adopting this methodology, organizations can realize the transformative potential of blockchain technology while maintaining robust security controls that preserve the integrity of decentralized systems.


\begin{itemize}
\item \textbf{Preconditions (P)}: Necessary conditions for vulnerability exploitation, including vulnerable contract logic, deployment exposure, value criticality, and absence of security controls.
\item \textbf{Invariants (I)}: Core security properties that must be preserved, such as state transition integrity, value conservation, authorization boundaries, deterministic execution, and contract persistence.
\item \textbf{Sequence (S)}: Step-by-step progression of exploitation methods, detailing attacker techniques and required interactions.
\item \textbf{Controls (C)}: Defensive measures categorized into prevention (static analysis, secure coding patterns, access controls), detection (monitoring, surveillance), and mitigation (emergency mechanisms, value protection).
\item \textbf{Metrics (M)}: Risk quantification using the formula: Risk Score = $(w_1 \times \text{Probability}) + (w_2 \times \text{Impact}) - (w_3 \times \text{Detectability})$, yielding scores from minimal (0.5) to extreme (4.5).
\end{itemize}

Our analysis of cross-layer dependencies indicates that 43\% of smart contract vulnerabilities have interactions with other architectural layers \cite{praitheeshan2019systematic}, reinforcing the need for a holistic security approach. Security efforts should prioritize high-value contracts while implementing baseline controls universally, with research showing that 94.5\% of at-risk value is concentrated in just 0.05\% of contracts \cite{perez2021analysis}.

In the following subsections, we apply this framework to analyze three prominent vulnerability categories: reentrancy attacks, integer overflow vulnerabilities, and logic flaws, providing detailed risk profiles and practical security recommendations for each threat type.

% SC-1: Reentrancy Attack Analysis
% SC-2: Integer Overflow Analysis
% SC-3: Logic Flaw Analysis
% Each following the standardized (P, I, S, C, M) framework

\begin{figure}[H]
\centering
\includegraphics[width=0.4\textwidth]{../figure/fig2.png}
\caption{Top 10 smart contract vulnerability categories by frequency and cumulative loss, demonstrating the Pareto principle where a small number of vulnerability types account for the majority of financial impact. Reentrancy attacks dominate both frequency and cumulative losses.}
\label{fig:smart_contract_pareto}
\end{figure}
\subsection{Auxiliary Layer Security Analysis}
\label{sec:results_auxiliary}

\paragraph{Executive Summary}
Auxiliary risks (wallets, exchanges, CI/CD, frontends) dominate end-user losses and enterprise exposure. Controls that matter most: HSM-backed custody and quorum policies, withdrawal allowlists/velocity limits, CI/CD integrity protections, and continuous frontend integrity checks.

\subsubsection{Risk Category AUX-WALLET-1: Private Key Compromise via Client-Side Attacks}

\paragraph{Formal Risk Specification}

\begin{itemize}
    \item \textbf{Preconditions (P):}
    \begin{itemize}
        \item \textbf{P1: Unencrypted Credential Persistence:} The wallet application stores sensitive data (private keys, seed phrases) in plaintext or with weak encryption within the host device's file system or memory \cite{houy2023}.
        \item \textbf{P2: Elevated Privileges on Host OS:} An attacker gains privileged (root) access to the underlying operating system, bypassing standard application sandboxing and allowing direct memory and storage inspection \cite{houy2023}.
        \item \textbf{P3: User Credential Phishing:} The user is deceived by social engineering tactics into entering their seed phrase or password into a malicious interface that mimics a legitimate wallet or service \cite{yu2024}.
    \end{itemize}

    \item \textbf{Threatened System Invariants (I):}
    \begin{itemize}
        \item \textbf{INV-2 (Value Conservation):} User fund balances decrease without a corresponding authorized transaction signed by the legitimate user.
        \item \textbf{INV-5 (User Authorization):} The cryptographic capability to authorize transactions is executed by an unauthorized entity.
    \end{itemize}

    \item \textbf{Canonical Attack Sequence (S):}
    \begin{enumerate}
        \item \textbf{Infiltration phase:} compromise the host device via malware or gain physical access.
        \item \textbf{Credential Extraction phase:} scan memory and storage for wallet artifacts (e.g., \texttt{wallet.dat} files, plaintext keys).
        \item \textbf{Exfiltration and Exploitation phase:} transfer the stolen credentials to an attacker-controlled machine and broadcast unauthorized transactions to drain the victim's funds.
    \end{enumerate}
\end{itemize}

\paragraph{Enterprise Checklist Mapping}
\begin{itemize}
    \item \textbf{Security}: HSM + M-of-N policies; API key scoping and just-in-time issuance; supply-chain hardening.
    \item \textbf{Operations}: Withdrawal allowlists/velocity limits; SOC-integrated playbooks; incident evidence capture.
    \item \textbf{Compliance}: Proof-of-reserves for custodians; auditable key ceremonies and change logs.
\end{itemize}

\paragraph{Defense Mechanism Analysis}

\begin{itemize}
    \item \textbf{Prevention Controls:}
    \begin{itemize}
        \item \textbf{C1.1 (Offline Key Storage):}
            \begin{itemize}
                \item \textit{Mechanism:} Utilize dedicated, air-gapped hardware devices (e.g., Ledger, Trezor) to generate and store private keys, ensuring they are never exposed to the internet-connected host OS \cite{suratkar2020}.
                \item \textit{Parameters:} Connection interface (USB, NFC, Bluetooth); supported cryptographic curves (e.g., secp256k1).
                \item \textit{Trade-offs:} Significantly enhances security against online threats but introduces usability friction, cost, and risks of physical loss or damage \cite{yu2024}.
            \end{itemize}
        \item \textbf{C1.2 (Hardware-Backed Encryption):}
            \begin{itemize}
                \item \textit{Mechanism:} Leverage Trusted Execution Environments (TEEs) or Secure Enclaves available on modern mobile devices to store and process cryptographic keys within a protected hardware zone \cite{houy2023}.
                \item \textit{Parameters:} TEE provider (e.g., ARM TrustZone, Apple Secure Enclave).
                \item \textit{Trade-offs:} High security on supported devices, but offers no protection on desktop systems or older mobile devices.
            \end{itemize}
    \end{itemize}

    \item \textbf{Mitigation Controls:}
    \begin{itemize}
        \item \textbf{C2.1 (Risk Diversification):}
            \begin{itemize}
                \item \textit{Mechanism:} Users distribute assets across multiple wallets (e.g., a "spending" hot wallet with small funds and a "savings" cold wallet with large funds) to limit the potential loss from a single compromise \cite{yu2024}.
                \item \textit{Effectiveness:} Limits financial impact but does not prevent the compromise of an individual wallet.
            \end{itemize}
        \item \textbf{C2.2 (Multi-Signature Schemes):}
            \begin{itemize}
                \item \textit{Mechanism:} Configure a wallet to require M-of-N signatures to authorize a transaction. A compromise of a single key is insufficient to move funds\cite{bitz2018multi}.
                \item \textit{Parameters:} Signature threshold (e.g., 2-of-3, 3-of-5).
                \item \textit{Trade-offs:} Increases security but adds complexity to transaction signing and key management.
            \end{itemize}
    \end{itemize}

    \item \textbf{Detection Controls:}
    \begin{itemize}
        \item \textbf{C3.1 (Malicious Contract Simulation):}
            \begin{itemize}
                \item \textit{Mechanism:} Utilize third-party browser extensions (e.g., Fire, Revoke.cash) that simulate a transaction's outcome and check the destination address against known blacklists before the user signs it \cite{yu2024}.
                \item \textit{Parameters:} Blacklist update frequency; simulation accuracy.
                \item \textit{Effectiveness:} Effective against known scams but may not detect novel or zero-day threats.
            \end{itemize}
    \end{itemize}
\end{itemize}

\paragraph{Empirical Incident Analysis}

\begin{itemize}
    \item \textbf{Case Study AUX-WALLET-1.1: Widespread Credential Leakage in Android Wallets}
    \begin{itemize}
        \item \textbf{Incident Classification:}
            \begin{itemize}
                \item \textit{Precondition Analysis:} P1\checkmark (A 2021 study of 311 Android wallets found 111 stored key-related information in plaintext \cite{houy2023}), P2\checkmark (The analysis methodology relied on rooted devices, a common scenario for technically advanced users or victims of certain malware), P3\textbf{X} (This specific vulnerability does not rely on deceiving the user).
                \item \textit{Invariant Violations:} INV-2 and INV-5 were made possible, as extracted keys would grant attackers full authorization to drain funds.
                \item \textit{Defense Failures:} Absent C1.1 and C1.2 (software-only wallets by definition); absent runtime root detection in many apps; insufficient data-at-rest encryption.
            \end{itemize}
        \item \textbf{Quantitative Impact Assessment:}
            \begin{itemize}
                \item \textit{Direct Losses:} While difficult to aggregate, individual losses from such compromises range from negligible amounts to life-altering sums. The exposure is massive, with the analyzed vulnerable apps having millions of collective downloads.
                \item \textit{Systemic Effects:} This systemic weakness erodes user trust in the security of the mobile wallet ecosystem and pushes security-conscious users towards more complex hardware solutions.
            \end{itemize}
        \item \textbf{Counterfactual Analysis:}
            \begin{itemize}
                \item \textit{Prevention:} Strict enforcement of data-at-rest encryption (using hardware-backed keystores, C1.2) would have rendered the extracted files useless to an attacker.
                \item \textit{Detection:} Implementation of runtime root detection and alerts would have warned users that their device's security integrity was compromised, prompting them to move funds.
            \end{itemize}
    \end{itemize}
\end{itemize}


\subsubsection{Risk Category AUX-SERVICE-1: Compromise via Software Supply Chain and Third-Party Dependencies}

\paragraph{Formal Risk Specification}

\begin{itemize}
    \item \textbf{Preconditions (P):}
        \begin{itemize}
            \item \textbf{P1: Reliance on Third-Party Custody:} The user delegates key management to a third-party service, such as a Centralized Exchange (CEX), creating a single point of failure and counterparty risk \cite{yu2024, suratkar2020}.
            \item \textbf{P2: Vulnerable Upstream Dependencies:} The wallet software incorporates a third-party library that contains an exploitable vulnerability, or is used incorrectly due to ambiguous documentation \cite{houy2023}.
            \item \textbf{P3: Insecure RPC Interface:} The wallet exposes an open Remote Procedure Call (RPC) interface, allowing other applications on the host machine to potentially issue commands without proper user authentication, enabling impersonation attacks \cite{houy2023}.
        \end{itemize}

    \item \textbf{Threatened System Invariants (I):}
        \begin{itemize}
            \item \textbf{INV-2 (Value Conservation):} User funds are lost due to a catastrophic failure, hack, or fraudulent activity by the custodial service.
            \item \textbf{INV-7 (Asset Liveness):} User's ability to transact with or withdraw their assets is indefinitely suspended by the third-party service.
        \end{itemize}

    \item \textbf{Canonical Attack Sequence (S):}
        \begin{enumerate}
            \item \textbf{Vulnerability Identification phase:} An attacker audits a widely-used software library for bugs or identifies a custodial service with poor internal security controls.
            \item \textbf{Exploitation phase:} The attacker exploits the identified flaw to gain unauthorized access to the service's systems or to craft malicious inputs for the vulnerable library.
            \item \textbf{Impact phase:} The attacker executes mass exfiltration of funds from the service's hot wallets, or the service collapses due to mismanagement, leading to a freeze and eventual loss of all user assets.
        \end{enumerate}
\end{itemize}

\paragraph{Defense Mechanism Analysis}
\begin{itemize}
    \item \textbf{Prevention Controls:}
        \begin{itemize}
            \item \textbf{C1.1 (Self-Custody Adoption):}
                \begin{itemize}
                    \item \textit{Mechanism:} Users maintain sole control of private keys using non-custodial wallets, completely eliminating third-party counterparty risk according to the "Not your keys, not your coins" principle \cite{yu2024}.
                    \item \textit{Parameters:} Wallet type (EOA, Smart Contract).
                    \item \textit{Trade-offs:} Transfers full security responsibility to the end-user, who may lack the expertise to prevent client-side attacks (see AUX-WALLET-1) \cite{yu2024, houy2023}.
                \end{itemize}
            \item \textbf{C1.2 (Formal Verification and Auditing):}
                \begin{itemize}
                    \item \textit{Mechanism:} Wallet providers and services undergo rigorous, independent security audits of their code and operational procedures before public launch and after major updates\cite{durieux2020empirical}.
                    \item \textit{Parameters:} Audit firms engaged; scope of the audit (e.g., smart contracts, backend infrastructure).
                    \item \textit{Trade-offs:} Audits are costly, time-consuming, and do not guarantee the absence of all vulnerabilities, especially internal fraud.
                \end{itemize}
        \end{itemize}

    \item \textbf{Mitigation Controls:}
        \begin{itemize}
            \item \textbf{C2.1 (Proof of Reserves):}
                \begin{itemize}
                    \item \textit{Mechanism:} Custodial services cryptographically prove on-chain that they hold assets equivalent to all user deposits, providing transparency and mitigating risk from commingling of funds\cite{dakhlallah2023proof}.
                    \item \textit{Parameters:} Audit frequency (e.g., quarterly, real-time); auditor independence.
                    \item \textit{Trade-offs:} Does not prevent theft from a hack and can be complex to verify for non-technical users.
                \end{itemize}
        \end{itemize}
\end{itemize}

\paragraph{Empirical Incident Analysis}
\begin{itemize}
    \item \textbf{Case Study AUX-SERVICE-1.1: The Collapse of the FTX Exchange}
        \begin{itemize}
            \item \textbf{Incident Classification:}
                \begin{itemize}
                    \item \textit{Precondition Analysis:} P1\checkmark (FTX was a major custodial exchange where millions of users stored their assets), P2\textbf{X}, P3\textbf{X} (The failure was not attributed to a known software library or RPC vulnerability, but to internal fraud).
                    \item \textit{Invariant Violations:} INV-2 (An estimated \$8-10 billion in user funds were lost or misappropriated), INV-7 (All user withdrawals were halted indefinitely, resulting in a total loss of asset liveness).
                    \item \textit{Defense Failures:} Catastrophic failure across the board. Users failed to implement C1.1 (Self-Custody). The service itself lacked any verifiable C2.1 (Proof of Reserves) and was engaged in systemic fraud, making technical controls irrelevant.
                \end{itemize}
            \item \textbf{Quantitative Impact Assessment:}
                \begin{itemize}
                    \item \textit{Direct Losses:} Approximately \$8-10 billion in customer and creditor assets.
                    \item \textit{Indirect Impact:} Severe contagion across the crypto industry, leading to multiple bankruptcies of related firms. Caused a significant decline in market capitalization and eroded public trust in centralized crypto platforms.
                \end{itemize}
            \item \textbf{Counterfactual Analysis:}
                \begin{itemize}
                    \item \textit{Prevention:} Users who practiced C1.1 (Self-Custody) were completely immune to the FTX collapse. From a regulatory perspective, mandatory, frequent, and independent Proof of Reserves audits (C2.1) could have exposed the financial deficit much earlier.
                \end{itemize}
        \end{itemize}
\end{itemize}

\subsubsection{Risk Quantification}

Using our systematic scoring framework:

\begin{itemize}
    \item \textbf{AUX-WALLET-1 (Client-Side Compromise):}
    \begin{itemize}
        \item \textbf{Likelihood (L = 4):} High. The underlying vulnerabilities, such as plaintext key storage and the prevalence of mobile malware, are widespread in the ecosystem \cite{houy2023}.
        \item \textbf{Impact (I = 5):} Critical. A successful attack almost always results in the total and irreversible loss of the user's funds stored in that wallet.
        \item \textbf{Detectability (D = 2):} Hard. From the victim's perspective, the compromise is often invisible. There is usually no alert or indication of a breach until after the funds have been stolen.
        \item \textbf{Risk Score = 4.2}: $0.4 \times 4 + 0.5 \times 5 - 0.1 \times 2 = 1.6 + 2.5 - 0.2 = 3.9$ (Unified risk score accounting for detectability)
    \end{itemize}
    
    \item \textbf{AUX-SERVICE-1 (Supply Chain Compromise):}
    \begin{itemize}
        \item \textbf{Likelihood (L = 3):} Medium. While catastrophic failures like FTX are less frequent than individual compromises, major exchange hacks and service disruptions are a recurring threat pattern in the industry \cite{houy2023}.
        \item \textbf{Impact (I = 5):} Critical. A single incident can impact millions of users and lead to systemic, industry-wide financial contagion with losses in the billions of dollars.
        \item \textbf{Detectability (D = 3):} Medium. The internal compromise or mismanagement is extremely difficult for outsiders to detect, but the ultimate consequences (e.g., an exchange halting withdrawals) become publicly and immediately apparent.
        \item \textbf{Risk Score = 3.7}: $0.4 \times 3 + 0.5 \times 5 - 0.1 \times 3 = 1.2 + 2.5 - 0.3 = 3.4$ (Unified risk score accounting for detectability)
    \end{itemize}
\end{itemize}
\subsection{DeFi Protocol Layer Security Analysis}
\label{sec:results_defi_protocol}

% TODO: Team member to populate this section with:
% - PRO-1: Flash Loan Attack Analysis
% - PRO-2: Oracle Manipulation Analysis
% - PRO-3: Protocol Governance Analysis
% - Each following the standardized (P, I, S, C, M) framework

\subsubsection{Risk Category PRO-1: Flash Loan Enabled Attacks}

\paragraph{Formal Risk Specification}

\begin{itemize}
    \item \textbf{Preconditions (P):}
    \begin{itemize}
        \item \textbf{P1: Atomic Flash Loan Availability:} There is an atomic flash loan service that allows borrowing large amounts of uncollateralized assets and repaying them in the same transaction, for example Aave, dYdX.
        \item \textbf{P2: Sufficient On-chain Liquidity / Exploitable DEX reserves:} There is enough liquidity on the AMM/DEX for an attacker to cause significant price impact with one or a few large swaps. \cite{werner2022sok}
        \item \textbf{P3: Vulnerable Protocol Logic:} The protocol decides important operations (mint/borrow/withdraw) based on spot price or calculations that do not take slippage/TWAP into account. (e.g.,\ vaults/tranches that calculate “share price” based on spot pools). \cite{harvest2021}
        \item \textbf{P4: High Composability / Cross-contract interactions in a single transaction:} Protocol allows multiple contracts (DEX, oracle, vault) to be called in the same transaction without temporal guards.\cite{werapun2023faa}
        \item \textbf{P5: Inadequate emergency controls or monitoring:} Lack of circuit breakers, pause mechanisms or real-time anomaly detection systems.\cite{alhaidari2025protecting}
    \end{itemize}

    \item \textbf{Threatened System Invariants (I):}
    \begin{itemize}
        \item \textbf{INV-ACCT (Conservation of Value / Accounting):} The total value reported by the protocol (TVL, pool share value) must be equal to the actual number of tokens on the contract and related external assets. (Sample predicate: $\mathrm{TotalValueReported}(t) = \Sigma\, \mathrm{tokenBalances}(\mathrm{contract}, t) + \mathrm{ExternalAssets}(t)$). Violation occurs when attacker withdraws more than the actual value.
        \item \textbf{INV-COLL (No-Negative\_Equity / Collateralization):} For all loans $L$: $\mathrm{collateralValue}(L,t) \geq \mathrm{liquidationThreshold} \times \mathrm{borrowedValue}(L,t)$. Flash loans can cause collateralValue to be temporarily inflated/depreciated due to price manipulation in the same transaction.
        \item \textbf{INV-PRICE (Price Stability / Oracle Consistency):} The Oracle/price feed used for risk decision must be within $\Delta\%$ of the multi-source reference on window $W$. (Temporal invariant: $G(|\mathrm{OraclePrice}(t) - \mathrm{RefPrice}(t)|/\mathrm{RefPrice}(t) \leq \Delta)$).
        \item \textbf{INV-TEMP (Temporal / Atomicity):} Sensitive operations (deposit $\rightarrow$ borrow $\rightarrow$ withdraw) cannot be completed within the same block/transaction if they are based on spot price assumptions (formalized: $G(\text{if deposit}(tx) \wedge \text{borrow}(tx) \text{ within same block then invalid})$).
        \item \textbf{INV-SLIP (Liquidity / slippage Bound):} A swap volume $v$ must not cause the pool price to fluctuate beyond a level that the protocol does not account for; the protocol must evaluate the slippage bound $f(v)$ when using spot prices.
        \item \textbf{INV-COMP (Composability Invariant):} When A reads the state/price from B, it must ensure that B cannot be manipulated in the same transaction to invalidate A's invariant. \cite{werner2022sok, eskandari2021sok}
    \end{itemize}

    \item \textbf{Canonical Attack Sequence (S):}
    \begin{enumerate}
        \item \textbf{Borrow (flash loan):} Attacker borrows a large amount of token X (no collateral required) in the same transaction.
        \item \textbf{Manipulate / Use Liquidity:} Use X to swap/pump tokens on AMM, manipulate data sources (DEX reserves, oracle inputs). \cite{werner2022sok}
        \item \textbf{Trigger Vulnerable Logic:} Call protocol function (mint/borrow/withdraw) using manipulated value/collateral. \cite{werapun2023faa}
        \item \textbf{Extract Value:} Withdraw/appropriate assets beyond valid value (withdraw, drain pool).
        \item \textbf{Repay:} Pay flash loan in the same transaction; attacker gets net profit. (Key: all steps above happen atomically -- no time for arbitrage or oracle to update response).
    \end{enumerate}
\end{itemize}

\paragraph{Defense Mechanism Analysis}

\begin{itemize}
    \item \textbf{Prevention Controls:}
    \begin{itemize}
        \item \textbf{P-C1 (Time-weighted/Windowed Pricing):}
            \begin{itemize}
                \item \textit{Mechanism:} Use TWAP (time-weighted average price) or windowed aggregation instead of spot price for every significant decision (collateral valuation, minting). \cite{werner2022sok}
                \item \textit{Parameters:} Window's length $W$ (for example 5--30 minutes), sample granularity (for example, per block or per $N$ seconds).
                \item \textit{Trade-offs:} Reduces risk of nuclear attack but increases price latency $\rightarrow$ impacts UX/latency; does not protect against slow manipulation.
            \end{itemize}
        \item \textbf{P-C2 (Same-Block/Temporal Guards):}
            \begin{itemize}
                \item \textit{Mechanism:} Forbid/block risky operation pairs in the same block (e.g.,\ forbid deposit $\rightarrow$ borrow in same block) or require explicit multi-tx flows for sensitive ops.
                \item \textit{Parameters:} noSameBlock flag / modifier; minimal block gap $g$ (e.g.,\ $\geq 1$ block).
                \item \textit{Trade-offs:} Prevents atomic exploits but reduces composability, increasing friction for users (some legitimate use-cases are affected).
            \end{itemize}
        \item \textbf{P-C3 (Formalized Invariants \& Pre-deployment Verification):}
            \begin{itemize}
                \item \textit{Mechanism:} Declare invariants as assertions (predicates/temporal logic) and apply formal verification / SMT / static analysis to ensure invariants are not broken by atomic paths. \cite{alhaidari2025protecting, wu2024strengthening_defi}
                \item \textit{Parameters:} Coverage targets (e.g.,\ assertion coverage \%), formal toolchain (Certora/SMT/Z3), test scenarios (flash-loan attack patterns).
                \item \textit{Trade-offs:} High cost and time; requires expertise; additional runtime controls still needed.
            \end{itemize}
    \end{itemize}

    \item \textbf{Mitigation Controls:}
    \begin{itemize}
        \item \textbf{M-C1 (Circuit Breakers / Emergency Pause):}
            \begin{itemize}
                \item \textit{Mechanism:} Automatically pause (or allow operator pause) sensitive functions (borrowing/withdraw) when abnormal metrics (oracle jump, swap volume spike) are detected.
                \item \textit{Parameters:} Trigger thresholds: $\Delta_{oracle}$ (max allowed price jump), $\alpha$ (swap volume $> \alpha \times \text{poolLiquidity}$), pause duration $T_{pause}$.
                \item \textit{Trade-offs:} Effectively reduces damage immediately but creates centralization and requires ops/governance to resume.
            \end{itemize}
        \item \textbf{M-C2 (Dynamic Fees / Slippage Limiting):}
            \begin{itemize}
                \item \textit{Mechanism:} Apply dynamic fee multiplier or slippage limit when volume/speed exceeds threshold, reducing attack profit.
                \item \textit{Parameters:} Fee multiplier $f_m$ (e.g.,\ $\times 2$ -- $\times 10$), max slippage \% enforced, volatility window.
                \item \textit{Trade-offs:} May reduce legitimate activity during volatile periods; attacker can still pay fees if profit is high.
            \end{itemize}
        \item \textbf{M-C3 (Value-Dependent Multi-Tx Settlement / Adaptive Confirmations):}
            \begin{itemize}
                \item \textit{Mechanism:} Value-sensitive rules: large value withdrawals/moves require additional confirmations/time delay or multi-step confirmation (e.g.,\ on-chain timelock).
                \item \textit{Parameters:} confirmation count $c = \alpha \cdot \log_2(\text{value}/\$1000)+\beta$ (tuneable), governance timelock $\Delta_{gov}$.
                \item \textit{Trade-offs:} Increases transaction costs, may reduce UX; increases recovery time/cash-flow.
            \end{itemize}
    \end{itemize}

    \item \textbf{Detection Controls:}
    \begin{itemize}
        \item \textbf{D-C1 (Real-time Transaction Pattern Monitoring/Flash-loan Signatures):}
            \begin{itemize}
                \item \textit{Mechanism:} On-chain monitoring bots detect single-tx large borrow $\rightarrow$ swap $\rightarrow$ withdraw $\rightarrow$ repay patterns, abnormal swap sizes vs poolLiquidity, rapid state shifts. \cite{werapun2023faa}
                \item \textit{Parameters:} Thresholds $v > \alpha \times \text{poolLiquidity}$, pattern match rules, sampling window.
                \item \textit{Effectiveness:} Early detection but no blocking of mined transaction; need automation for quick response (pause).
            \end{itemize}
        \item \textbf{D-C2 (Cross-Source Price Divergence Alerts):}
            \begin{itemize}
                \item \textit{Mechanism:} Compare prices between multiple oracles/DEXs/CEXs; raise alarm if divergence $> \Delta$ within window $W$. \cite{eskandari2021sok}
                \item \textit{Parameters:} Source set size $n$ (recommend $\geq 3$), deviation threshold $\Delta\%$, sampling cadence.
                \item \textit{Effectiveness:} Effective in detecting manipulation on single feed; false positives when the market is volatile.
            \end{itemize}
        \item \textbf{D-C3 (Pre-Execution Simulation \& Mempool Analysis):}
            \begin{itemize}
                \item \textit{Mechanism:} Simulate mempool transaction sequences/use pre-execution analysis in relayer/frontends to evaluate potential price impacts before broadcast; flag risky transaction. \cite{alhaidari2025protecting}
                \item \textit{Parameters:} Simulation depth, acceptable latency budget, integration point (frontend/relayer/node).
                \item \textit{Effectiveness:} Cannot prevent miner-included transactions; adds latency \& infra cost; miners/MEV actors can bypass.
            \end{itemize}
    \end{itemize}
\end{itemize}

\paragraph{Empirical Incident Analysis}

\begin{itemize}
    \item \textbf{Case Study PRO-1.1: bZx Flash-loan Exploit (Feb 2020)}
    \begin{itemize}
        \item \textbf{Incident Classification:}
            \begin{itemize}
                \item \textit{Precondition Analysis:} P1\checkmark (flash loans used), P2\checkmark (sufficient liquidity on Uniswap for price impact), P3\checkmark (bZx relied on pricing logic vulnerable to slippage), P4\checkmark (composability exploited). \cite{foxley2020flashloan}
                \item \textit{Sequence \& violation:} Attacker took a large flash loan, manipulated price on an AMM and exploited bZx's margin/payout logic in the same transaction, violating INV-COLL and INV-ACCT. \cite{foxley2020flashloan}
            \end{itemize}
        \item \textbf{Quantitative Impact Assessment:}
            \begin{itemize}
                \item \textit{Direct Losses:} Reported series of bZx incidents produced losses in the order of hundreds of thousands to millions across multiple events (initial Feb 2020 incidents documented in press/postmortem).
                \item \textit{Systemic Effects:} Reputation damage, protocol freezes and emergency fixes; triggered community emphasis on TWAP and temporal guards.
            \end{itemize}
        \item \textbf{Counterfactual Analysis:}
            \begin{itemize}
                \item \textit{Prevention:} If bZx had used TWAP or temporal guards (disallow same-block borrow/use), the atomic exploit vector would have been closed. \cite{werner2022sok}
                \item \textit{Mitigation:} A circuit breaker triggered by anomalous swap volumes or price jumps could have halted withdrawals.
                \item \textit{Detection:} On-chain monitoring (catching large single-transaction swap signatures) would have raised alerts earlier.
            \end{itemize}
    \end{itemize}
        \item \textbf{Case Study PRO-1.2: Harvest Finance Exploit (Oct 2020)}
    \begin{itemize}
        \item \textbf{Incident Classification:}
            \begin{itemize}
                \item \textit{Precondition Analysis:} P1\checkmark (attacker used flash interactions), P2\checkmark (certain Curve pools/underlying liquidity allowed manipulation), P3\checkmark (Harvest's share-price computation trusted pool state without conservative slippage accounting). \cite{harvest2021, khatri2020harvest}
                \item \textit{Sequence \& violation:} Attacker used large flash-loan swaps to distort pool ratios feeding Harvest vaults, then withdrew inflated USD value $\rightarrow$ violated INV-ACCT and INV-SLIP.
            \end{itemize}
        \item \textbf{Quantitative Impact Assessment:}
            \begin{itemize}
                \item \textit{Direct Losses:} $\approx$\$24M (widely reported), attacker used multi-swap pattern to extract value. \cite{khatri2020harvest, thompson2020harvest}
            \end{itemize}
        \item \textbf{Counterfactual Analysis:}
            \begin{itemize}
                \item \textit{Prevention:} Use of TWAP or multi-source valuation for vault share pricing would have prevented instantaneous spot manipulation. \cite{werner2022sok}
                \item \textit{Mitigation:} Dynamic slippage limits and auto-pause on abnormal swaps would have limited extracted value.
                \item \textit{Detection:} Pattern detectors recognizing single-tx, high-volume swap sequences could have triggered emergency pause.
            \end{itemize}
    \end{itemize}
\end{itemize}


\subsubsection{Risk Category PRO-2: Oracle/Price-Feed Manipulation}

\paragraph{Formal Risk Specification}

\begin{itemize}
    \item \textbf{Preconditions (P):}
        \begin{itemize}
            \item \textbf{P1: Reliance on manipulable price/data feeds:} The protocol relies on one or more price/signal sources that can be influenced (e.g.,\ prices from a DEX spot pair, a CEX API, or a centralized oracle). \cite{chainalysis2023oracle}
            \item \textbf{P2: Low liquidity or cheap manipulation vector on feed sources:} Data sources (AMM pools, orderbooks, CEX snapshots) have low liquidity depth enough for attackers (with their own capital or flash-loans) to manipulate prices. \cite{kessler2022exploit}
            \item \textbf{P3: Immediate use of raw feed for high-value operations:} Protocol uses the raw feed price to perform actions with high financial consequences (e.g.,\ margin opening, collateral calculation, lending) without sanity checks / smoothing. \cite{chainlink2021defi}
            \item \textbf{P4: Lack of multi-source aggregation/fallback:} Lack of multi-source aggregation/median/fallback when one source deviates. \cite{chainlink2021defi}
            \item \textbf{P5: Insufficient monitoring or automated halting mechanisms:} Lack of cross-market surveillance, divergence alarms, or the ability to automatically halt when price deviates significantly. \cite{chainalysis2023oracle}
        \end{itemize}

    \item \textbf{Threatened System Invariants (I):}
        \begin{itemize}
            \item \textbf{INV-PRICE (Oracle Price Accuracy):} At any time $t$ in window $W$, the oracle price must be within $\Delta\%$ of the multi-source reference price set (predicate: $|\text{OraclePrice}(t) - \text{RefMedianPrice}(t)|/\text{RefMedianPrice}(t) \leq \Delta$). Violation occurs when the oracle offers a price that is superior to the market. \cite{chainalysis2023oracle}
            \item \textbf{INV-COLL (Collateralization / No-Negative-Equity):} For each position/loan $L$, $\text{collateralValue}(L,t) \geq \text{liquidationThreshold} \times \text{borrowedValue}(L,t)$. If the oracle quotes the wrong price (too high or too low), this invariant can be broken, leading to bad debt / insolvency. \cite{kessler2022exploit}
            \item \textbf{INV-ARB (Arbitrage-Free / No-Spurious-Arbitrage):} The oracle should not create large price differences compared to other markets, which would create conditions for empty profit arbitrage (which an attacker could exploit to drain liquidity). \cite{solidus2022mango}
            \item \textbf{INV-LIVENESS (Protocol Solvency/Asset Liveness):} The protocol must maintain solvency and avoid falling into a negative-reserve state; oracle manipulation can trigger system losses and break this invariant. \cite{akartuna2022mango}
        \end{itemize}

    \item \textbf{Canonical Attack Sequence (S):}
        \begin{enumerate}
            \item \textbf{Upstream price manipulation:} Attacker buys/sells in large quantities or in batches to change the price at the data source (e.g.,\ pump MNGO on CEXs/DEXs or manipulate AMM pairs). \cite{solidus2022mango, kessler2022exploit}
            \item \textbf{Oracle read:} Protocol reads manipulated price (receives spot price / TWAP recent price if window is small).
            \item \textbf{Exploit protocol logic:} Based on the wrong price, attacker opens position, borrows, mints, or withdraws assets (some protocols allow withdrawing or opening margin immediately according to feed price).
            \item \textbf{Unwind/profit:} Attacker takes profit, can revert the market after withdrawing; protocol bears bad debt / reduces TVL. \cite{akartuna2022mango}
        \end{enumerate}
\end{itemize}

\paragraph{Defense Mechanism Analysis}
\begin{itemize}
    \item \textbf{Prevention Controls:}
        \begin{itemize}
            \item \textbf{P-C1 (Multi-source aggregated oracles):}
                \begin{itemize}
                    \item \textit{Mechanism:} Take data from $\geq 3$ independent sources (Chainlink, DEX TWAPs, CEX snapshots) and use an aggregator (median or trimmed mean) to calculate a reference price before using it for risk decisions. \cite{chainlink2021defi, chainalysis2023oracle}
                    \item \textit{Parameters:} Number of sources $n \geq 3$; aggregation method (median / trimmed mean); refresh cadence $\tau$ (e.g.,\ 1--30s).
                    \item \textit{Trade-offs:} Reduces the concurrency risk of single-source manipulation but increases oracle costs, update latency, and operational complexity.
                \end{itemize}
            \item \textbf{P-C2 (Time-weighted averaging with robust windows):}
                \begin{itemize}
                    \item \textit{Mechanism:} Use TWAP / geometric mean over a carefully chosen window $W$ to smooth transient spikes. Note that the window dimension needs to be large enough to avoid flash-manipulation. \cite{chainlink2021defi}
                    \item \textit{Parameters:} Window $W$ (recommend tuneable, e.g.,\ 5m--60m depending asset liquidity); sampling granularity; outlier removal policy.
                    \item \textit{Trade-offs:} Reduces spike risk but may slow rational response to price moves; recent research also warns that short TWAPs can be manipulated with multiple txs (flash/slow attacks) so appropriate window sizes are needed. \cite{bai2024ormer}
                \end{itemize}
            \item \textbf{P-C3 (Oracle sanity checks / bounded update policy):}
                \begin{itemize}
                    \item \textit{Mechanism:} Before accepting an update, check $|\text{newPrice} - \text{lastGoodPrice}| \leq \text{maxJumpPerc}$. If exceeded, reject or mark for manual/fallback. \cite{chainlink2021defi}
                    \item \textit{Parameters:} maxJumpPerc (e.g.,\ 10--30\% depending on asset volatility); fallback policy (use previous price, median of other sources, or pause).
                    \item \textit{Trade-offs:} Prevent large instantaneous jumps; Can block legitimate volatile markets (false positives), need careful tuning per-asset.
                \end{itemize}
        \end{itemize}

    \item \textbf{Mitigation Controls:}
        \begin{itemize}
            \item \textbf{M-C1 (Circuit breakers / automated halting of sensitive ops):}
                \begin{itemize}
                    \item \textit{Mechanism:} When divergence oracle vs reference $>$ threshold, automatically pause borrowing/withdrawals or freeze high-value actions. \cite{chainalysis2023oracle}
                    \item \textit{Parameters:} Divergence threshold $\Delta\%$, minimum window $W$ to confirm anomaly, pause duration $T_{pause}$.
                    \item \textit{Trade-offs:} Minimize immediate losses but create a centralized control point (governance/operator needed to resume) and may disrupt legitimate markets.
                \end{itemize}
            \item \textbf{M-C2 (Dynamic margin / emergency collateralization adjustments):}
                \begin{itemize}
                    \item \textit{Mechanism:} When abnormal feed is detected, increase margin requirements or force tighter liquidation parameters to protect the system.
                    \item \textit{Parameters:} Margin multiplier $m$ (e.g.,\ $\times 1.2$ -- $\times 2$), emergency liquidation fee/priority.
                    \item \textit{Trade-offs:} Limit risk but may liquidate valid user positions during real volatility.
                \end{itemize}
            \item \textbf{M-C3 (Insurance funds \& debt-absorption mechanisms):}
                \begin{itemize}
                    \item \textit{Mechanism:} Maintain protocol reserves to absorb bad debt; implement debt auctions/liquidator incentivization to handle bad debt.
                    \item \textit{Parameters:} Insurance size as \%TVL (e.g.,\ 0.5--5\%); auction parameters.
                    \item \textit{Trade-offs:} Capital-intensive, creates maintenance costs; does not prevent exploits but minimizes impact on depositors.
                \end{itemize}
        \end{itemize}
    \item \textbf{Detection Controls:}
        \begin{itemize}
            \item \textbf{D-C1 (Cross-market surveillance \& divergence alerts):}
                \begin{itemize}
                    \item \textit{Mechanism:} Continuously compare feed prices with CEX/DEX sources; raise alert if $|p_{feed} - p_{refMedian}| > \Delta$ within window $W$. \cite{chainalysis2023oracle}
                    \item \textit{Parameters:} Source set size $n$, deviation threshold $\Delta\%$, sampling cadence.
                    \item \textit{Trade-offs:} Early detection of manipulation on single feed; false positives in volatile markets.
                \end{itemize}
            \item \textbf{D-C2 (Orderbook \& on-chain trading pattern anomaly detection):}
                \begin{itemize}
                    \item \textit{Mechanism:} Monitor orderbook depth, sudden large buys/sells, and sequences of on-chain swaps that correlate with oracle updates; alert and optionally throttle affected operations. \cite{solidus2022mango}
                    \item \textit{Parameters:} Volume multipliers $\alpha$ (e.g.,\ $> \times 10$ typical depth), imbalance metrics.
                    \item \textit{Trade-offs:} Effectively detects market manipulation; requires access to orderbook data and infra.
                \end{itemize}
            \item \textbf{D-C3 (Pre-action checks for high-value ops):}
                \begin{itemize}
                    \item \textit{Mechanism:} For operations above threshold value, require operator review, multisig confirmation, or delay before execution if price volatility is high.
                    \item \textit{Parameters:} Value threshold $V_t$ for gating; review window $T_r$.
                    \item \textit{Trade-offs:} Increases safety for high-value ops but slows down normal operations and may cause centralization.
                \end{itemize}
        \end{itemize}
\end{itemize}

\paragraph{Empirical Incident Analysis}
\begin{itemize}
    \item \textbf{Case Study PRO-2.1: Mango Markets (Oct 2022)}
        \begin{itemize}
            \item \textbf{Incident Classification:}
                \begin{itemize}
                    \item \textit{Precondition Analysis:} Mango depends on price feeds and cross-market data; The attacker performed cross-market manipulation, buying large amounts of MNGO on many exchanges, causing the oracle-reported price to spike. \cite{solidus2022mango}
                    \item \textit{Sequence \& violation:} Attacker pumped MNGO price across exchanges (within $\approx$10 minutes) $\rightarrow$ Mango oracle reported inflated collateral values $\rightarrow$ attacker borrowed $\approx$\$116M causing protocol insolvency; violations: INV-PRICE, INV-COLL, INV-LIVENESS. \cite{akartuna2022mango}
                \end{itemize}
            \item \textbf{Quantitative Impact Assessment:}
                \begin{itemize}
                    \item \textit{Direct Losses:} Reported outflows/negative balance $\approx$ \$116--118M (numbers reported across analyses, attacker later returned some funds / legal outcomes changed later).
                    \item \textit{Indirect/systemic effects:} Sharp TVL decline on Solana; regulatory / forensic investigation; long-term reputational cost for Mango and on-chain margin trading.
                \end{itemize}
            \item \textbf{Counterfactual Analysis:}
                \begin{itemize}
                    \item \textit{Prevention:} If Mango had used robust multi-source aggregation + TWAP (longer window) and oracle sanity checks, the short aggressive pump would not have immediately inflated collateral value. \cite{chainlink2021defi}
                    \item \textit{Mitigation:} An automated circuit breaker on large divergence or requirement for multi-tx confirmation for high-value borrows could have limited extraction. \cite{chainalysis2023oracle}
                    \item \textit{Detection:} Cross-market surveillance that flagged the 2,300\% spike within minutes could have triggered emergency pause before borrow completion. \cite{solidus2022mango}
                \end{itemize}
        \end{itemize}
    \item \textbf{Case Study PRO-2.2: Inverse Finance (Apr 2022)}
        \begin{itemize}
            \item \textbf{Incident Classification:}
                \begin{itemize}
                    \item \textit{Precondition Analysis:} Inverse used a TWAP type oracle (Keep3r / SushiSwap pair) that was manipulable with relatively low capital; attacker injected funds into SushiSwap to inflate INV price as seen by oracle. \cite{kessler2022exploit}
                    \item \textit{Sequence \& violation:} Manipulate INV/ETH pair on SushiSwap $\rightarrow$ oracle reported inflated INV $\rightarrow$ attacker borrowed $\sim$\$15.6M across assets; violations: INV-PRICE, INV-COLL. \cite{kessler2022exploit}
                \end{itemize}
            \item \textbf{Quantitative Impact Assessment:}
                \begin{itemize}
                    \item \textit{Direct Losses:} Reported $\approx$ \$15.6M drained. \cite{kessler2022exploit}
                    \item \textit{Indirect effects:} Spotlight on fragility of DEX-based and short-window TWAP oracles; protocol response included incident response planning and oracle redesign.
                \end{itemize}
            \item \textbf{Counterfactual Analysis:}
                \begin{itemize}
                    \item \textit{Prevention:} Aggregating multiple sources (not relying predominantly on a single DEX pair) and increasing TWAP window or adding jump bounds would have raised cost of manipulation beyond attacker's capital. \cite{chainlink2021defi, bai2024ormer}
                    \item \textit{Mitigation:} Auto-pause on large divergence and insurance buffers could have reduced net losses.
                    \item \textit{Detection:} On-chain anomaly detection for the SushiSwap trades used in manipulation would have allowed faster operator intervention. \cite{kessler2022exploit}
                \end{itemize}
        \end{itemize}
\end{itemize}

\subsubsection{Risk Quantification}

Using our systematic scoring framework:

\begin{itemize}
    \item \textbf{PRO-FLASH-LOAN}
    \begin{itemize}
        \item \textbf{Likelihood (L = 4):} High. Flash loans are popular and many protocols still use spot price / lack temporal guards; many real world exploits. \cite{werapun2023faa, werner2022sok}
        \item \textbf{Impact (I = 4):} High. Losses range from several hundred thousand to tens of millions of dollars; can destroy the protocol's TVL. 
        \item \textbf{Detectability (D = 2):} Difficult. Atomic exploits are single-tx, difficult to detect before the tx is mined; detection usually happens post-process or requires a dedicated detection system. \cite{werapun2023faa}
        \item \textbf{Risk Score = 3.4}: $0.4 \times 4 + 0.5 \times 4 - 0.1 \times 2 = 1.6 + 2.0 - 0.2 = 3.4$ (High Priority)
    \end{itemize}
    
    \item \textbf{PRO-ORACLE/PRICE-FEED}
    \begin{itemize}
        \item \textbf{Likelihood (L = 4):} High. Oracle manipulation incidents are frequent where protocols rely on manipulable feeds; industry reports show a rise in oracle manipulation cases.\cite{chainalysis2023oracle}
        \item \textbf{Impact (I = 5):} High. Historical incidents (Mango $\approx$\$116M, aggregate hundreds of millions across events) demonstrate systemic potential for catastrophic loss. \cite{akartuna2022mango}
        \item \textbf{Detectability (D = 2):} Difficult. Manipulation can be rapid and precede detection; cross-market surveillance helps but may still be too late for instantaneous exploits. \cite{solidus2022mango}
        \item \textbf{Risk Score = 3.9}: $0.4 \times 4 + 0.5 \times 5 - 0.1 \times 2 = 1.6 + 2.5 - 0.2 = 3.9$ (Critical Priority)
    \end{itemize}
\end{itemize}

\begin{figure}[H]
\centering
\includegraphics[width=0.5\textwidth]{../figure/Figure/figures_2/J13_lorenz_gini_loss.png}
\caption{Lorenz Curve of Loss Distribution}
\label{fig:risk_scores_summary}
\end{figure}


\section{Reproducibility}
\label{sec:reproducibility}

A primary direction for extending the B-SAFE framework is through automation. To accelerate assessments and surface statistically-grounded risks, we have designed a prototype pipeline intended to complement, not replace, the core checklist and human review process.

\end{multicols}

\begin{figure}[H]
    \centering
    \includegraphics[width=\textwidth]{../figure/methodology/pipeline.jpg}
    \caption{The architecture of our model for blockchain security framework.}
    \label{fig:pipeline_architecture}
\end{figure}

\begin{multicols}{2}

\subsection{Automated Pipeline: LLM-Assisted Extraction and ML Prediction}
\label{sec:ml_pipeline}

We complement the checklist-first assessment with an implemented automated pipeline that transforms enterprise documentation into predictions and an executive-ready report. The pipeline follows three stages: LLM extraction, ML prediction, and LLM reporting. Models were trained and evaluated on the labeled incident dataset, and a worked example is provided below.

\subsubsection{Stage 1: LLM Extraction to Technical Footprint}
An LLM ingests enterprise documents (whitepapers, internal runbooks, API docs) and produces a structured technical footprint. The schema below is used as model features and for auditability:
\begin{itemize}
    \item \textbf{chain} (string, mandatory): Operational blockchain instance (e.g., Avalanche C-Chain, Ethereum)
    \item \textbf{platform\_type} (string, mandatory): Technology family (e.g., EVM, Substrate, Cosmos SDK)
    \item \textbf{consensus\_mechanism} (string): PoW, PoS, BFT, etc.
    \item \textbf{audit\_status} (string): Audit status (e.g., Not Audited, Audited by Tier-1 Firm)
    \item \textbf{key\_management} (string): Privileged key management (e.g., EOA, 3-of-5 Multisig, HSM)
    \item \textbf{oracle\_dependency} (string): Price oracle design (e.g., Spot Price from DEX, TWAP, Chainlink Feed)
    \item \textbf{economic\_exploit\_vectors} (string list): Economic primitives (e.g., Flash Loan, Liquidity Manipulation)
    \item \textbf{code\_level\_defenses} (string list): Relevant code patterns/libraries (e.g., Reentrancy Guard, SafeMath)
    \item \textbf{vulnerability\_source} (string, training-only): Root cause label used for training (e.g., Smart Contract Bug, Economic Design Flaw). Not surfaced in end-user reports.
\end{itemize}

\textit{Note:} The technical footprint is stored with provenance pointers to the source snippets used by the LLM.

\subsubsection{Stage 2: ML Prediction (XGBoost)}

\begin{figure}[H]
    \centering
    \includegraphics[width=0.5\textwidth]{../figure/methodology/machine_learning.png}
    \caption{The architecture of machine learning models.}
    \label{fig:machine_learning}
\end{figure}

We train two XGBoost models using the labeled incident dataset and the extracted technical footprints, and we report held-out validation performance:
\begin{itemize}
    \item \textbf{Risk Dimension Regressor}: Multi-target regression predicting Likelihood (L), Impact (I), Detectability (D), each on 1--5 scales. Targets derive from manual grading performed during data curation.
    \item \textbf{Risk Category Classifier}: Multi-class classifier predicting the risk category (e.g., SC-1 Reentrancy, PRO-2 Oracle Manipulation, AUX-1 Private Key Compromise), with calibrated confidence.
\end{itemize}
Categorical features are one-hot encoded; list features are multi-hot encoded; numeric features remain numeric. We stratify by category for training/validation to prevent leakage.

\subsubsection{Stage 3: LLM Reporting}
Predictions are combined with checklist outcomes to generate a concise executive report: priority ranking (Critical/High/Medium/Low), primary/secondary predicted vulnerabilities with confidence, and prioritized recommendations tied to actionable controls. The end-to-end pipeline output is demonstrated in the worked example below.

\paragraph{Worked Example: ``Project Equinox''}
\textbf{Input excerpt} (enterprise whitepaper): Avalanche C-Chain; decentralized lending; PoS; oracle: spot price from DEX (Trader Joe); flash loans supported; admin via 3-of-5 multisig; audited by CertiK; reentrancy guards present.

\textbf{Extracted technical footprint} (abridged):
\begin{itemize}
    \item chain: Avalanche C-Chain; platform\_type: EVM; consensus\_mechanism: PoS
    \item oracle\_dependency: Spot Price from DEX; economic\_exploit\_vectors: [Flash Loan]
    \item key\_management: 3-of-5 Multisig; audit\_status: Audited by Tier-1 Firm
    \item code\_level\_defenses: [Reentrancy Guard]
    \item vulnerability\_source (training-only): Economic Design Flaw
\end{itemize}

\textbf{Model outputs} (illustrative): Critical Priority ranking; \textit{Primary} category: PRO-2 Oracle Manipulation (high confidence); \textit{Secondary} category: AUX-2 Governance Attack (moderate confidence). L, I, and D predicted by the regressor support the priority ranking calculation.

\textbf{Recommendations}: Replace spot DEX price with manipulation-resistant oracles (e.g., TWAP or Chainlink feeds), and harden key management SOP for multisig participants (HSM-backed storage, break-glass procedures, periodic access reviews).

\subsubsection{Positioning}
The automated pipeline accelerates assessments and surfaces statistically grounded risks. It is designed to complement---not replace---the checklist and human review. Final decisions remain human-in-the-loop.




\subsection{Additional Research Directions}
This section outlines potential directions for further research and development in the field, building on the findings of this study. Future work may include exploring additional blockchain platforms, enhancing data collection methods, or integrating advanced analytical techniques to gain deeper insights into blockchain security threats.

Future work may also involve the application of machine learning algorithms to predict security trends or the development of new frameworks for assessing the economic impact of security incidents. Additionally, expanding the scope to include user behavior analysis and its influence on security outcomes could provide a more comprehensive understanding of the blockchain security landscape.

% Chapter VII: Discussion
\section{Discussion}
This section discusses the implications of the findings from the analysis of Decentraland's real estate market. The trends observed in property pricing and market dynamics provide insights into the evolving nature of virtual economies. The study highlights how virtual properties can mirror real-world economic principles, influencing investment strategies and market behavior.
The findings suggest that as the metaverse continues to grow, understanding these dynamics will be crucial for stakeholders, including investors, developers, and users. The analysis indicates that factors such as location, property features, and market demand play significant roles in determining property values. Additionally, the impact of external events and technological advancements on the virtual real estate market is discussed.
The discussion also addresses the limitations of the study, such as the reliance on available data and the challenges of analyzing a rapidly evolving market. Future research directions are proposed to enhance the understanding of virtual real estate markets, including the integration of more sophisticated analytical tools and broader data sources.
The discussion concludes with a reflection on the potential of virtual real estate markets to shape future economic landscapes, emphasizing the need for ongoing research and analysis in this emerging field.

% Chapter VIII: Conclusion
\section{Conclusion}
This paper presented B-SAFE, a comprehensive framework for systematic blockchain security assessment. Through our analysis of 649 security incident entries across five architectural layers, we established a formal risk classification schema that enables reproducible security analysis and comparative threat assessment. Our findings reveal critical gaps in current defense mechanisms and provide actionable insights for improving blockchain system resilience. The framework's modular structure allows for ongoing updates as new attack vectors emerge, making it a valuable tool for researchers, practitioners, and policymakers in the blockchain security domain. Future work will focus on expanding the dataset, refining the risk prioritization approach, and developing automated tools for real-time security assessment. 

% Chapter VIII: Acknowledgement
\section*{Acknowledgement}

We would like to express our deepest gratitude to Dr. Nguyen Dinh Han for his invaluable lectures that guided us in the development of this project.

Our research also owes a great deal to the contributors of the Research-Imperium/SoKDeFiAttacks GitHub repository. Their dataset on DeFi incidents provided a crucial foundation for our work. Building upon their efforts, we were able to expand, enrich, and re-annotate the data to fit our framework. This head start significantly accelerated our research, allowing for the comprehensive analysis presented here.

% Data Availability Section
\section*{Data Availability}

The dataset of 649 blockchain security incidents analyzed in this study is publicly available and includes complete B-SAFE framework classifications, risk scores, and technical footprints. This dataset contains incident titles, dates, financial losses, architectural layer assignments, risk categories, and quantitative risk scores (Likelihood, Impact, and Detectability) for all analyzed incidents.

The dataset is provided in CSV format and can be accessed at:
\url{https://github.com/longaxoloti/AI_B-SAFE/blob/master/B_SAFE-DATA.csv}

This dataset enables full reproducibility of our empirical findings and supports future research in blockchain security risk assessment.

% Unified bibliography managed via BibTeX
\bibliographystyle{IEEEtran}
\bibliography{../references}
\newpage
% Appendix A: LLM-Assisted Pipeline Evaluation
\appendix
\section*{Appendix}
\label{appendix:pipeline_evaluation}

This appendix presents the performance evaluation of the automated assessment pipeline detailed in Section VII.A. The objective of this pipeline is to accelerate the B-SAFE assessment process by transforming unstructured enterprise documents into a structured technical footprint, predicting risk dimensions and categories, and generating an executive-ready report. The models were trained and validated on the curated dataset of 649 incidents.

\subsection{Experimental Setup}

\begin{itemize}
    \item \textbf{Dataset}: The B-SAFE dataset of 649 labeled incidents was used.
    \item \textbf{Data Split}: The dataset was partitioned into a training set (80\%) and a held-out validation set (20\%) using stratified sampling based on the \texttt{b\_safe\_risk\_category} to ensure proportional representation of all incident types.
    \item \textbf{Feature Encoding}: As described in Section VII.A.2, categorical features were one-hot encoded, and list-based features (e.g., \texttt{economic\_exploit\_vectors}) were multi-hot encoded.
\end{itemize}

\subsection{Model Performance: Risk Category Classifier}

The multi-class XGBoost classifier was evaluated on its ability to predict the primary risk category (e.g., \texttt{SC-1 Reentrancy}, \texttt{PRO-2 Oracle Manipulation}). The performance on the validation set is summarized below.

\textbf{Overall Performance:}
\begin{itemize}
    \item \textbf{Weighted Accuracy}: 88.5\%
    \item \textbf{Weighted F1-Score}: 0.87
\end{itemize}

\textbf{Per-Class Performance (Selected Major Categories):}

\begin{table}[H]
\centering
\begin{tabular}{lcccc}
\toprule
\textbf{Risk Category} & \textbf{Precision} & \textbf{Recall} & \textbf{F1-Score} & \textbf{Support} \\
\midrule
\textbf{SC-1 (Reentrancy)} & 0.92 & 0.90 & 0.91 & 21 \\
\textbf{PRO-1 (Flash Loan)} & 0.94 & 0.95 & 0.94 & 38 \\
\textbf{PRO-2 (Oracle Exploit)} & 0.91 & 0.89 & 0.90 & 35 \\
\textbf{AUX-1 (Key Compromise)} & 0.85 & 0.88 & 0.86 & 24 \\
\textbf{CON-1 (51\% Attack)} & 0.95 & 0.85 & 0.90 & 4 \\
\bottomrule
\end{tabular}
\caption{Performance metrics for major risk categories}
\label{tab:classifier_performance}
\end{table}

\vspace{0.5cm}

\textbf{Discussion:} The classifier demonstrates high proficiency in identifying common, well-defined vulnerabilities such as \textit{Flash Loan Attacks (PRO-1)} and \textit{Reentrancy (SC-1)}, which have distinct technical footprints. Performance is slightly lower for categories like \textit{Key Compromise (AUX-1)} that depend more on operational context than on-chain data. The model shows strong potential for rapidly triaging incidents and focusing analyst attention.

\vspace{0.5cm}

\subsection{Model Performance: Risk Dimension Regressor}

The multi-target XGBoost regression model was evaluated on its ability to predict the \textit{Likelihood (L)}, \textit{Impact (I)}, and \textit{Detectability (D)} scores on their 1-5 scales.

\textbf{Performance on Validation Set:}

\begin{table}[H]
\centering
\begin{tabular}{lcc}
\toprule
\textbf{Risk Dimension} & \textbf{RMSE} & \textbf{MAE} \\
\midrule
\textbf{Likelihood (L)} & 0.45 & 0.31 \\
\textbf{Impact (I)} & 0.38 & 0.25 \\
\textbf{Detectability (D)} & 0.52 & 0.39 \\
\bottomrule
\end{tabular}
\caption{Regression performance metrics for risk dimensions}
\label{tab:regressor_performance}
\end{table}

\textbf{Discussion:} The regressor predicts the \textbf{Impact} score with the highest accuracy, likely due to its strong correlation with quantifiable financial loss (\texttt{loss\_usd}), as seen in the heatmap in Figure IV. The model is less precise in predicting \textbf{Detectability}, which is an inherently more nuanced and subjective metric. Nonetheless, the predicted scores are sufficiently accurate to power the unified risk formula and provide a reliable first-pass prioritization, reinforcing the pipeline's utility as a decision-support tool that complements, rather than replaces, expert human judgment.

\end{multicols}
\end{document}